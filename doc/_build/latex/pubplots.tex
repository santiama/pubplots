% Generated by Sphinx.
\def\sphinxdocclass{report}
\newif\ifsphinxKeepOldNames \sphinxKeepOldNamestrue
\documentclass[letterpaper,10pt,english]{sphinxmanual}
\usepackage{iftex}

\ifPDFTeX
  \usepackage[utf8]{inputenc}
\fi
\ifdefined\DeclareUnicodeCharacter
  \DeclareUnicodeCharacter{00A0}{\nobreakspace}
\fi
\usepackage{cmap}
\usepackage[T1]{fontenc}
\usepackage{amsmath,amssymb,amstext}
\usepackage{babel}
\usepackage{times}
\usepackage[Sonny]{fncychap}
\usepackage{longtable}
\usepackage{sphinx}
\usepackage{multirow}
\usepackage{eqparbox}


\addto\captionsenglish{\renewcommand{\figurename}{Fig.\@ }}
\addto\captionsenglish{\renewcommand{\tablename}{Table }}
\SetupFloatingEnvironment{literal-block}{name=Listing }

\addto\extrasenglish{\def\pageautorefname{page}}

\setcounter{tocdepth}{1}


\title{pubplots Documentation}
\date{Oct 18, 2016}
\release{0.1.dev}
\author{Brendan Bulfin}
\newcommand{\sphinxlogo}{}
\renewcommand{\releasename}{Release}
\makeindex

\makeatletter
\def\PYG@reset{\let\PYG@it=\relax \let\PYG@bf=\relax%
    \let\PYG@ul=\relax \let\PYG@tc=\relax%
    \let\PYG@bc=\relax \let\PYG@ff=\relax}
\def\PYG@tok#1{\csname PYG@tok@#1\endcsname}
\def\PYG@toks#1+{\ifx\relax#1\empty\else%
    \PYG@tok{#1}\expandafter\PYG@toks\fi}
\def\PYG@do#1{\PYG@bc{\PYG@tc{\PYG@ul{%
    \PYG@it{\PYG@bf{\PYG@ff{#1}}}}}}}
\def\PYG#1#2{\PYG@reset\PYG@toks#1+\relax+\PYG@do{#2}}

\expandafter\def\csname PYG@tok@gd\endcsname{\def\PYG@tc##1{\textcolor[rgb]{0.63,0.00,0.00}{##1}}}
\expandafter\def\csname PYG@tok@gu\endcsname{\let\PYG@bf=\textbf\def\PYG@tc##1{\textcolor[rgb]{0.50,0.00,0.50}{##1}}}
\expandafter\def\csname PYG@tok@gt\endcsname{\def\PYG@tc##1{\textcolor[rgb]{0.00,0.27,0.87}{##1}}}
\expandafter\def\csname PYG@tok@gs\endcsname{\let\PYG@bf=\textbf}
\expandafter\def\csname PYG@tok@gr\endcsname{\def\PYG@tc##1{\textcolor[rgb]{1.00,0.00,0.00}{##1}}}
\expandafter\def\csname PYG@tok@cm\endcsname{\let\PYG@it=\textit\def\PYG@tc##1{\textcolor[rgb]{0.25,0.50,0.56}{##1}}}
\expandafter\def\csname PYG@tok@vg\endcsname{\def\PYG@tc##1{\textcolor[rgb]{0.73,0.38,0.84}{##1}}}
\expandafter\def\csname PYG@tok@vi\endcsname{\def\PYG@tc##1{\textcolor[rgb]{0.73,0.38,0.84}{##1}}}
\expandafter\def\csname PYG@tok@mh\endcsname{\def\PYG@tc##1{\textcolor[rgb]{0.13,0.50,0.31}{##1}}}
\expandafter\def\csname PYG@tok@cs\endcsname{\def\PYG@tc##1{\textcolor[rgb]{0.25,0.50,0.56}{##1}}\def\PYG@bc##1{\setlength{\fboxsep}{0pt}\colorbox[rgb]{1.00,0.94,0.94}{\strut ##1}}}
\expandafter\def\csname PYG@tok@ge\endcsname{\let\PYG@it=\textit}
\expandafter\def\csname PYG@tok@vc\endcsname{\def\PYG@tc##1{\textcolor[rgb]{0.73,0.38,0.84}{##1}}}
\expandafter\def\csname PYG@tok@il\endcsname{\def\PYG@tc##1{\textcolor[rgb]{0.13,0.50,0.31}{##1}}}
\expandafter\def\csname PYG@tok@go\endcsname{\def\PYG@tc##1{\textcolor[rgb]{0.20,0.20,0.20}{##1}}}
\expandafter\def\csname PYG@tok@cp\endcsname{\def\PYG@tc##1{\textcolor[rgb]{0.00,0.44,0.13}{##1}}}
\expandafter\def\csname PYG@tok@gi\endcsname{\def\PYG@tc##1{\textcolor[rgb]{0.00,0.63,0.00}{##1}}}
\expandafter\def\csname PYG@tok@gh\endcsname{\let\PYG@bf=\textbf\def\PYG@tc##1{\textcolor[rgb]{0.00,0.00,0.50}{##1}}}
\expandafter\def\csname PYG@tok@ni\endcsname{\let\PYG@bf=\textbf\def\PYG@tc##1{\textcolor[rgb]{0.84,0.33,0.22}{##1}}}
\expandafter\def\csname PYG@tok@nl\endcsname{\let\PYG@bf=\textbf\def\PYG@tc##1{\textcolor[rgb]{0.00,0.13,0.44}{##1}}}
\expandafter\def\csname PYG@tok@nn\endcsname{\let\PYG@bf=\textbf\def\PYG@tc##1{\textcolor[rgb]{0.05,0.52,0.71}{##1}}}
\expandafter\def\csname PYG@tok@no\endcsname{\def\PYG@tc##1{\textcolor[rgb]{0.38,0.68,0.84}{##1}}}
\expandafter\def\csname PYG@tok@na\endcsname{\def\PYG@tc##1{\textcolor[rgb]{0.25,0.44,0.63}{##1}}}
\expandafter\def\csname PYG@tok@nb\endcsname{\def\PYG@tc##1{\textcolor[rgb]{0.00,0.44,0.13}{##1}}}
\expandafter\def\csname PYG@tok@nc\endcsname{\let\PYG@bf=\textbf\def\PYG@tc##1{\textcolor[rgb]{0.05,0.52,0.71}{##1}}}
\expandafter\def\csname PYG@tok@nd\endcsname{\let\PYG@bf=\textbf\def\PYG@tc##1{\textcolor[rgb]{0.33,0.33,0.33}{##1}}}
\expandafter\def\csname PYG@tok@ne\endcsname{\def\PYG@tc##1{\textcolor[rgb]{0.00,0.44,0.13}{##1}}}
\expandafter\def\csname PYG@tok@nf\endcsname{\def\PYG@tc##1{\textcolor[rgb]{0.02,0.16,0.49}{##1}}}
\expandafter\def\csname PYG@tok@si\endcsname{\let\PYG@it=\textit\def\PYG@tc##1{\textcolor[rgb]{0.44,0.63,0.82}{##1}}}
\expandafter\def\csname PYG@tok@s2\endcsname{\def\PYG@tc##1{\textcolor[rgb]{0.25,0.44,0.63}{##1}}}
\expandafter\def\csname PYG@tok@nt\endcsname{\let\PYG@bf=\textbf\def\PYG@tc##1{\textcolor[rgb]{0.02,0.16,0.45}{##1}}}
\expandafter\def\csname PYG@tok@nv\endcsname{\def\PYG@tc##1{\textcolor[rgb]{0.73,0.38,0.84}{##1}}}
\expandafter\def\csname PYG@tok@s1\endcsname{\def\PYG@tc##1{\textcolor[rgb]{0.25,0.44,0.63}{##1}}}
\expandafter\def\csname PYG@tok@ch\endcsname{\let\PYG@it=\textit\def\PYG@tc##1{\textcolor[rgb]{0.25,0.50,0.56}{##1}}}
\expandafter\def\csname PYG@tok@m\endcsname{\def\PYG@tc##1{\textcolor[rgb]{0.13,0.50,0.31}{##1}}}
\expandafter\def\csname PYG@tok@gp\endcsname{\let\PYG@bf=\textbf\def\PYG@tc##1{\textcolor[rgb]{0.78,0.36,0.04}{##1}}}
\expandafter\def\csname PYG@tok@sh\endcsname{\def\PYG@tc##1{\textcolor[rgb]{0.25,0.44,0.63}{##1}}}
\expandafter\def\csname PYG@tok@ow\endcsname{\let\PYG@bf=\textbf\def\PYG@tc##1{\textcolor[rgb]{0.00,0.44,0.13}{##1}}}
\expandafter\def\csname PYG@tok@sx\endcsname{\def\PYG@tc##1{\textcolor[rgb]{0.78,0.36,0.04}{##1}}}
\expandafter\def\csname PYG@tok@bp\endcsname{\def\PYG@tc##1{\textcolor[rgb]{0.00,0.44,0.13}{##1}}}
\expandafter\def\csname PYG@tok@c1\endcsname{\let\PYG@it=\textit\def\PYG@tc##1{\textcolor[rgb]{0.25,0.50,0.56}{##1}}}
\expandafter\def\csname PYG@tok@o\endcsname{\def\PYG@tc##1{\textcolor[rgb]{0.40,0.40,0.40}{##1}}}
\expandafter\def\csname PYG@tok@kc\endcsname{\let\PYG@bf=\textbf\def\PYG@tc##1{\textcolor[rgb]{0.00,0.44,0.13}{##1}}}
\expandafter\def\csname PYG@tok@c\endcsname{\let\PYG@it=\textit\def\PYG@tc##1{\textcolor[rgb]{0.25,0.50,0.56}{##1}}}
\expandafter\def\csname PYG@tok@mf\endcsname{\def\PYG@tc##1{\textcolor[rgb]{0.13,0.50,0.31}{##1}}}
\expandafter\def\csname PYG@tok@err\endcsname{\def\PYG@bc##1{\setlength{\fboxsep}{0pt}\fcolorbox[rgb]{1.00,0.00,0.00}{1,1,1}{\strut ##1}}}
\expandafter\def\csname PYG@tok@mb\endcsname{\def\PYG@tc##1{\textcolor[rgb]{0.13,0.50,0.31}{##1}}}
\expandafter\def\csname PYG@tok@ss\endcsname{\def\PYG@tc##1{\textcolor[rgb]{0.32,0.47,0.09}{##1}}}
\expandafter\def\csname PYG@tok@sr\endcsname{\def\PYG@tc##1{\textcolor[rgb]{0.14,0.33,0.53}{##1}}}
\expandafter\def\csname PYG@tok@mo\endcsname{\def\PYG@tc##1{\textcolor[rgb]{0.13,0.50,0.31}{##1}}}
\expandafter\def\csname PYG@tok@kd\endcsname{\let\PYG@bf=\textbf\def\PYG@tc##1{\textcolor[rgb]{0.00,0.44,0.13}{##1}}}
\expandafter\def\csname PYG@tok@mi\endcsname{\def\PYG@tc##1{\textcolor[rgb]{0.13,0.50,0.31}{##1}}}
\expandafter\def\csname PYG@tok@kn\endcsname{\let\PYG@bf=\textbf\def\PYG@tc##1{\textcolor[rgb]{0.00,0.44,0.13}{##1}}}
\expandafter\def\csname PYG@tok@cpf\endcsname{\let\PYG@it=\textit\def\PYG@tc##1{\textcolor[rgb]{0.25,0.50,0.56}{##1}}}
\expandafter\def\csname PYG@tok@kr\endcsname{\let\PYG@bf=\textbf\def\PYG@tc##1{\textcolor[rgb]{0.00,0.44,0.13}{##1}}}
\expandafter\def\csname PYG@tok@s\endcsname{\def\PYG@tc##1{\textcolor[rgb]{0.25,0.44,0.63}{##1}}}
\expandafter\def\csname PYG@tok@kp\endcsname{\def\PYG@tc##1{\textcolor[rgb]{0.00,0.44,0.13}{##1}}}
\expandafter\def\csname PYG@tok@w\endcsname{\def\PYG@tc##1{\textcolor[rgb]{0.73,0.73,0.73}{##1}}}
\expandafter\def\csname PYG@tok@kt\endcsname{\def\PYG@tc##1{\textcolor[rgb]{0.56,0.13,0.00}{##1}}}
\expandafter\def\csname PYG@tok@sc\endcsname{\def\PYG@tc##1{\textcolor[rgb]{0.25,0.44,0.63}{##1}}}
\expandafter\def\csname PYG@tok@sb\endcsname{\def\PYG@tc##1{\textcolor[rgb]{0.25,0.44,0.63}{##1}}}
\expandafter\def\csname PYG@tok@k\endcsname{\let\PYG@bf=\textbf\def\PYG@tc##1{\textcolor[rgb]{0.00,0.44,0.13}{##1}}}
\expandafter\def\csname PYG@tok@se\endcsname{\let\PYG@bf=\textbf\def\PYG@tc##1{\textcolor[rgb]{0.25,0.44,0.63}{##1}}}
\expandafter\def\csname PYG@tok@sd\endcsname{\let\PYG@it=\textit\def\PYG@tc##1{\textcolor[rgb]{0.25,0.44,0.63}{##1}}}

\def\PYGZbs{\char`\\}
\def\PYGZus{\char`\_}
\def\PYGZob{\char`\{}
\def\PYGZcb{\char`\}}
\def\PYGZca{\char`\^}
\def\PYGZam{\char`\&}
\def\PYGZlt{\char`\<}
\def\PYGZgt{\char`\>}
\def\PYGZsh{\char`\#}
\def\PYGZpc{\char`\%}
\def\PYGZdl{\char`\$}
\def\PYGZhy{\char`\-}
\def\PYGZsq{\char`\'}
\def\PYGZdq{\char`\"}
\def\PYGZti{\char`\~}
% for compatibility with earlier versions
\def\PYGZat{@}
\def\PYGZlb{[}
\def\PYGZrb{]}
\makeatother

\renewcommand\PYGZsq{\textquotesingle}

\begin{document}

\maketitle
\tableofcontents
\phantomsection\label{index::doc}


Contents:


\chapter{pubplots}
\label{modules:welcome-to-pubplots-s-documentation}\label{modules:pubplots}\label{modules::doc}

\section{pubplots package}
\label{pubplots:pubplots-package}\label{pubplots::doc}

\subsection{Submodules}
\label{pubplots:submodules}

\subsection{pubplots PlotData class}
\label{pubplots:pubplots-plotdata-class}\index{PlotData (class in pubplots.plotdata)}

\begin{fulllineitems}
\phantomsection\label{pubplots:pubplots.plotdata.PlotData}\pysigline{\sphinxstrong{class }\sphinxcode{pubplots.plotdata.}\sphinxbfcode{PlotData}}
used to store information from csv files for plotting. It can smartly set the axes labels
and line labels using the file headers
\paragraph{Attributes}

\noindent\begin{tabulary}{\linewidth}{|L|L|}
\hline

files
&
(list of loaded files)
\\
\hline
fits
&
(list of fits of the plot data)
\\
\hline
frames
&
(list of pandas DataFrames that the data is taken from)
\\
\hline
xaxislabel
&
(label to be put on the xaxis)
\\
\hline
yaxislabel
&
(label to be put on the yaxis)
\\
\hline
yset
&
(list of data to be plotted {[}{[}x array,y array{]}, {[}x2 array, y2 array{]}.....{]})
\\
\hline
labels
&
(list of labels for the yset)
\\
\hline
yerrors
&
(list of yerror arrays to go with yset)
\\
\hline
yrset
&
(list of data to be plotted to right hand axes {[}{[}x array,y array{]}, {[}x2 array, y2 array{]}.....{]})
\\
\hline
yraxislabel
&
(right hand axes label)
\\
\hline
yrlabels
&
(list of labels for the right hand axes data)
\\
\hline
yr2set
&
(list of data to be plotted to right hand axes {[}{[}x array,y array{]}, {[}x2 array, y2 array{]}.....{]})
\\
\hline
yr2axislabel
&
(second right hand axes label)
\\
\hline
yr2labels
&
(second right hand line labels)
\\
\hline\end{tabulary}

\paragraph{Methods}
\index{filelist() (pubplots.plotdata.PlotData method)}

\begin{fulllineitems}
\phantomsection\label{pubplots:pubplots.plotdata.PlotData.filelist}\pysiglinewithargsret{\sphinxbfcode{filelist}}{\emph{files={[}{]}, header=0, xcol=0, ycols={[}1{]}, labels={[}{]}, yrcols={[}{]}, yrlabels={[}{]}, yr2cols={[}{]}, yr2labels={[}{]}, xerrors={[}{]}, yerrors={[}{]}, yrerrors={[}{]}, yr2errors={[}{]}, xaxislabel=None, yaxislabel=None, yraxislabel=None, yr2axislabel=None, **kwargs}}{}
Load a list of files, pass {\color{red}\bfseries{}**}kwargs to pandas.read\_csv
\begin{quote}\begin{description}
\item[{Parameters}] \leavevmode
\textbf{files} : list of strings, optional
\begin{quote}

list of files to be loaded
\end{quote}

\textbf{header} : int, optional
\begin{quote}

Which row to use as a header, default is header=0 which takes the first row
\end{quote}

\textbf{other paramaters:}
\begin{quote}

see PlotData.prepare\_frame method
\end{quote}

\textbf{**kwargs} : TYPE
\begin{quote}

aditional arguements passed to pandas.read\_csv method
\end{quote}

\end{description}\end{quote}

\end{fulllineitems}

\index{fit() (pubplots.plotdata.PlotData method)}

\begin{fulllineitems}
\phantomsection\label{pubplots:pubplots.plotdata.PlotData.fit}\pysiglinewithargsret{\sphinxbfcode{fit}}{\emph{deg=1}}{}
fit the data usying a polynomial, store a fit of the data and the errors in self.fits.
It also prints to the terminal the fit paramaters and errors
\begin{quote}\begin{description}
\item[{Parameters}] \leavevmode
\textbf{deg} : int, optional
\begin{quote}

The degree of the polynomial to be fit.
\end{quote}

\end{description}\end{quote}

\end{fulllineitems}

\index{onefile() (pubplots.plotdata.PlotData method)}

\begin{fulllineitems}
\phantomsection\label{pubplots:pubplots.plotdata.PlotData.onefile}\pysiglinewithargsret{\sphinxbfcode{onefile}}{\emph{filename, header=0, xcol=0, ycols={[}1{]}, labels={[}{]}, yrcols={[}{]}, yrlabels={[}{]}, yr2cols={[}{]}, yr2labels={[}{]}, xerrors={[}{]}, yerrors={[}{]}, yrerrors={[}{]}, yr2errors={[}{]}, xaxislabel=None, yaxislabel=None, yraxislabel=None, yr2axislabel=None, **kwargs}}{}
Add data from a single file to our data set, pass {\color{red}\bfseries{}**}kwargs to pandas.read\_csv and then
uses prepare frame
\begin{quote}\begin{description}
\item[{Parameters}] \leavevmode
\textbf{filename} : str
\begin{quote}

The file to be processed
\end{quote}

\textbf{header} : int, optional
\begin{quote}

Which row to use as a header, default is header=0 which takes the first row
\end{quote}

\textbf{other paramaters:}
\begin{quote}

see PlotData.prepare\_frame method
\end{quote}

\textbf{**kwargs} : TYPE
\begin{quote}

aditional arguements passed to pandas.read\_csv method
\end{quote}

\end{description}\end{quote}

\end{fulllineitems}

\index{prepare\_frame() (pubplots.plotdata.PlotData method)}

\begin{fulllineitems}
\phantomsection\label{pubplots:pubplots.plotdata.PlotData.prepare_frame}\pysiglinewithargsret{\sphinxbfcode{prepare\_frame}}{\emph{dataframe, xcol=0, ycols={[}1{]}, labels={[}{]}, yrcols={[}{]}, yrlabels={[}{]}, yr2cols={[}{]}, yr2labels={[}{]}, xerrors={[}{]}, yerrors={[}{]}, xaxislabel=None, yaxislabel=None, yraxislabel=None, yr2axislabel=None}}{}
Specifiy what data is what in the pandas data. This essentially builds list of
pointers to access the correct data for plotting
\begin{quote}\begin{description}
\item[{Parameters}] \leavevmode
\textbf{dataframe} : pandas.DataFrame object to set the plotting data from

\textbf{xcol} : int, optional
\begin{quote}

position of the xdata collumn
\end{quote}

\textbf{ycols} : list of integers, optional
\begin{quote}

list of collumns to use for ydata
\end{quote}

\textbf{labels} : list, optional
\begin{quote}

list of labels for the ycols data
\end{quote}

\textbf{yrcols} : list, optional
\begin{quote}

list of collumns to be plotted to a right hand axes
\end{quote}

\textbf{yrlabels} : list, optional
\begin{quote}

list of labels to go with the right hand axes data
\end{quote}

\textbf{yr2cols} : list of integers, optional
\begin{quote}

list of columns to be plot to second right hand axes
\end{quote}

\textbf{yr2labels} : list, optional
\begin{quote}

labels to go with second right hand axes
\end{quote}

\textbf{xerrors} : list of integers, optional

\textbf{yerrors} : list of integers, optional
\begin{quote}

list of collumns to use as the yerrors for the yset data
\end{quote}

\textbf{xaxislabel} : str, optional

\textbf{yaxislabel} : str, optional

\textbf{yraxislabel} : str, optional

\textbf{yr2axislabel} : str, optional
\begin{quote}

Description
\end{quote}

\end{description}\end{quote}

\end{fulllineitems}

\index{smooth() (pubplots.plotdata.PlotData method)}

\begin{fulllineitems}
\phantomsection\label{pubplots:pubplots.plotdata.PlotData.smooth}\pysiglinewithargsret{\sphinxbfcode{smooth}}{\emph{window\_len=5}, \emph{window='blackman'}}{}
smooth all of the yset data. Window length must be an odd number. By default it uses
the blackman window which is 0 on the end. That means a value of window\_len = 5 or greater
is required to actualy do some smoothing. This removes the pointer to the original data
and replaces it with numpy arrays of the smoothed data
\begin{quote}\begin{description}
\item[{Parameters}] \leavevmode
\textbf{x: the input signal}

\textbf{window\_len: int optional,}
\begin{quote}

the dimension of the smoothing window; should be an odd integer
\end{quote}

\textbf{window: str optional}
\begin{quote}

the type of window from `flat', `hanning', `hamming', `bartlett', `blackman'
flat window will produce a moving average smoothing. default is blackman
\end{quote}

\end{description}\end{quote}

\end{fulllineitems}

\index{walkandfind() (pubplots.plotdata.PlotData method)}

\begin{fulllineitems}
\phantomsection\label{pubplots:pubplots.plotdata.PlotData.walkandfind}\pysiglinewithargsret{\sphinxbfcode{walkandfind}}{\emph{startpath='data', search=None, header=0, xcol=0, ycols={[}1{]}, labels={[}{]}, yrcols={[}{]}, yrlabels={[}{]}, yr2cols={[}{]}, yr2labels={[}{]}, xerrors={[}{]}, yerrors={[}{]}, xaxislabel=None, yaxislabel=None, yraxislabel=None, yr2axislabel=None, **kwargs}}{}
Search in a specified path for files containing a certain string and then load them
up as data.The path can be relative or an absolute path.
\begin{quote}\begin{description}
\item[{Parameters}] \leavevmode
\textbf{startpath} : str, optional
\begin{quote}

Description
\end{quote}

\textbf{search} : str, optional
\begin{quote}

load file names containing this string, default is `.csv'.
\end{quote}

\textbf{header} : int, optional
\begin{quote}

Which row to use as a header, default is header=0 which takes the first row
\end{quote}

\textbf{other paramaters:}
\begin{quote}

see PlotData.prepare\_frame method
\end{quote}

\textbf{**kwargs} : TYPE
\begin{quote}

aditional arguements passed to pandas.read\_csv method
\end{quote}

\end{description}\end{quote}

\end{fulllineitems}


\end{fulllineitems}



\subsection{pubplots.plot module}
\label{pubplots:module-pubplots.plot}\label{pubplots:pubplots-plot-module}\index{pubplots.plot (module)}
A set of scripts for producing nice plots using matplotlib,
Made to be used in combination with the class PlotData, which contains methods
for selecting the data from csv files and assiging axes labels and data labels.

Note this is not a wrapper to replace matplolib, it is just a set of scripts
which take matplotlib axes objects and make nicer plots than the default matplotlib ones.
\index{add\_yerrors() (in module pubplots.plot)}

\begin{fulllineitems}
\phantomsection\label{pubplots:pubplots.plot.add_yerrors}\pysiglinewithargsret{\sphinxcode{pubplots.plot.}\sphinxbfcode{add\_yerrors}}{\emph{ax}, \emph{yset=None}, \emph{errors=None}, \emph{colors='tb10'}, \emph{elinewidth=1.5}}{}
Add x or y or both error bars to the plot
\begin{quote}\begin{description}
\item[{Parameters}] \leavevmode
\textbf{ax} : matplotlib.axes object

\textbf{yset} : list
\begin{quote}

list of data to plot like{[}{[}x1array, y1array{]}, {[}x2array, y2array{]}.....{]}.
x1,y1 are arrays or lists of numbers for plotting
\end{quote}

\textbf{errors} : list
\begin{quote}

list of yerrors to go with each data set in yset
\end{quote}

\textbf{colors} : str or list of (r,g,b) tupples
\begin{quote}

Pass a string options are `black', `grey', `tb10', `tb20' and `cb10'(for colorblind people)
\end{quote}

\textbf{elinewidth} : float, optional

\end{description}\end{quote}

\end{fulllineitems}

\index{axis\_labels() (in module pubplots.plot)}

\begin{fulllineitems}
\phantomsection\label{pubplots:pubplots.plot.axis_labels}\pysiglinewithargsret{\sphinxcode{pubplots.plot.}\sphinxbfcode{axis\_labels}}{\emph{ax}, \emph{xaxislabel='x'}, \emph{yaxislabel='y'}, \emph{title=None}, \emph{fontsize=20}}{}
Lable the x a nd y axis and optionally add a title.
\begin{quote}\begin{description}
\item[{Parameters}] \leavevmode
\textbf{ax} : matplotlib.axes object
\begin{quote}

this is the maplotlib axes to be labeled
\end{quote}

\textbf{xaxislabel} : str, optional

\textbf{yaxislabel} : str, optional

\textbf{title} : None, optional
\begin{quote}

add title str if you want a title
\end{quote}

\textbf{fontsize} : int, optional

\end{description}\end{quote}

\end{fulllineitems}

\index{inset\_plot() (in module pubplots.plot)}

\begin{fulllineitems}
\phantomsection\label{pubplots:pubplots.plot.inset_plot}\pysiglinewithargsret{\sphinxcode{pubplots.plot.}\sphinxbfcode{inset\_plot}}{\emph{fig, ax, yset, lbwh={[}0.58, 0.58, 0.4, 0.4{]}, grid=False, dashes=None, xlabel=None, ylabel=None, title=None, fontsize=14, colors='tb10', style='modern', scatter=False, label=False, labels={[}{]}, at\_x=None, linestyles='-`, markers='var', **kwargs}}{}
Make an inset plot positioned in the top right
\begin{quote}\begin{description}
\item[{Parameters}] \leavevmode
\textbf{fig} : matplotlib.figure object
\begin{quote}

The figure to add the inset to
\end{quote}

\textbf{ax} : matplotlib.axes object

\textbf{yset} : list
\begin{quote}

list of data to plot like{[}{[}x1array, y1array{]}, {[}x2array, y2array{]}.....{]}.
x1,y1 are arrays or pointers to arrays/lists of numbers
\end{quote}

\textbf{lbwh} : list, optional
\begin{quote}

{[}l,b,w,h{]}, l, b is where to place the inset and w, h is its size
\end{quote}

\textbf{grid} : bool, optional
\begin{quote}

add a grid, default False
\end{quote}

\textbf{dashes} : None, optional
\begin{quote}

True - adds carying dashes
\end{quote}

\textbf{xlabel} : None or str, optional
\begin{quote}

str to add to the xaxis label
\end{quote}

\textbf{ylabel} : None or str, optional
\begin{quote}

string fro yaxis label
\end{quote}

\textbf{title} : None or str, optionl
\begin{quote}

string for inset title
\end{quote}

\textbf{fontsize} : int, optional

\textbf{colors} : str or list of (r,g,b) tupples
\begin{quote}

Pass a string options are `black', `grey', `tb10', `tb20' and `cb10'(for colorblind people)
\end{quote}

\textbf{style} : str, optional
\begin{quote}

one of `modern', `semimodern' or `oldhat'
\end{quote}

\textbf{scatter} : bool, optional
\begin{quote}

use markers instead of lines
\end{quote}

\textbf{label} : bool, optional
\begin{quote}

label the lines
\end{quote}

\textbf{labels} : list, optional
\begin{quote}

list of labels
\end{quote}

\textbf{at\_x} : None or list, optional
\begin{quote}

list of x co-cordinates to align the labels with
\end{quote}

\textbf{linestyles} : TYPE, optional

\textbf{markers} : str, optional

\textbf{**kwargs} : TYPEV
\begin{quote}

passed to matplotlib axes.plot()
\end{quote}

\item[{Returns}] \leavevmode
matplotlib.axes object
\begin{quote}

returns the new right hand axes
\end{quote}

\end{description}\end{quote}

\end{fulllineitems}

\index{label\_line() (in module pubplots.plot)}

\begin{fulllineitems}
\phantomsection\label{pubplots:pubplots.plot.label_line}\pysiglinewithargsret{\sphinxcode{pubplots.plot.}\sphinxbfcode{label\_line}}{\emph{ax}, \emph{x}, \emph{y}, \emph{label\_text}, \emph{color}, \emph{at\_x=0}, \emph{rotation\_on=False}, \emph{fontsize=14}, \emph{offset=(0}, \emph{0)}}{}
This Function allows to put labels onto a line graph. the labesl land on the line by
by default. The can be set to rotate to be inline with the line. roation\_on = True. There
can also be an ofset. It must also be passed an axis. Typically ax in my code.
\begin{quote}\begin{description}
\item[{Parameters}] \leavevmode
\textbf{ax} : matplotlib.axes object

\textbf{x} : array
\begin{quote}

the xdata
\end{quote}

\textbf{y} : array
\begin{quote}

the ydata
\end{quote}

\textbf{label\_text} : str

\textbf{color} : (r,g,b) tuple or standard matplotlib color

\textbf{at\_x} : int, optional
\begin{quote}

x psoition for label
\end{quote}

\textbf{rotation\_on} : bool, optional

\textbf{fontsize} : int, optional

\textbf{offset} : tuple, optional
\begin{quote}

(x,y) offset from intial selected position
\end{quote}

\end{description}\end{quote}

\end{fulllineitems}

\index{label\_lines() (in module pubplots.plot)}

\begin{fulllineitems}
\phantomsection\label{pubplots:pubplots.plot.label_lines}\pysiglinewithargsret{\sphinxcode{pubplots.plot.}\sphinxbfcode{label\_lines}}{\emph{ax, yset, at\_x=None, rotation\_on=False, labels={[}{]}, offsets={[}(0, 0), (0, 0), (0, 0), (0, 0), (0, 0), (0, 0), (0, 0), (0, 0), (0, 0), (0, 0){]}, colors='tb10', fontsize=18}}{}
Add in graph labels, which are often much better than having a legend. Uses np.interpolate
together with the yset's to place the label
\begin{quote}\begin{description}
\item[{Parameters}] \leavevmode
\textbf{ax} : matplotlib.axes object

\textbf{yset} : list
\begin{quote}

list of data to plot like{[}{[}x1array, y1array{]}, {[}x2array, y2array{]}.....{]}.
x1,y1 are arrays or lists of numbers for plotting
\end{quote}

\textbf{at\_x} : None or list, optional
\begin{quote}

list of x co-cordinates to align the labels with
\end{quote}

\textbf{rotation\_on} : bool, optional
\begin{quote}

rotate the labels inline with the plotted data
\end{quote}

\textbf{labels} : list, optional
\begin{quote}

list of labels
\end{quote}

\textbf{offsets} : TYPE, optional
\begin{quote}

list of offsets to move the data from the intial placement point
\end{quote}

\textbf{colors} : str or list of (r,g,b) tupples
\begin{quote}

Pass a string options are `black', `grey', `tb10', `tb20' and `cb10'(for colorblind people)
\end{quote}

\textbf{fontsize} : int, optional

\end{description}\end{quote}

\end{fulllineitems}

\index{modern\_style() (in module pubplots.plot)}

\begin{fulllineitems}
\phantomsection\label{pubplots:pubplots.plot.modern_style}\pysiglinewithargsret{\sphinxcode{pubplots.plot.}\sphinxbfcode{modern\_style}}{\emph{ax}, \emph{fontsize=18}, \emph{grid=True}}{}
modern style. No boundary and y-grid. ticks on left and bottom
\begin{quote}\begin{description}
\item[{Parameters}] \leavevmode
\textbf{ax} : matplotlib.axes object

\textbf{fontsize} : int, optional

\textbf{grid} : bool, optional

\end{description}\end{quote}

\end{fulllineitems}

\index{old\_hat\_style() (in module pubplots.plot)}

\begin{fulllineitems}
\phantomsection\label{pubplots:pubplots.plot.old_hat_style}\pysiglinewithargsret{\sphinxcode{pubplots.plot.}\sphinxbfcode{old\_hat\_style}}{\emph{ax}, \emph{fontsize=18}}{}
Typical graph style. Bounding box. Black lines
\begin{quote}\begin{description}
\item[{Parameters}] \leavevmode
\textbf{ax} : matplotlib.axes object

\textbf{fontsize} : int, optional

\end{description}\end{quote}

\end{fulllineitems}

\index{plot\_lines() (in module pubplots.plot)}

\begin{fulllineitems}
\phantomsection\label{pubplots:pubplots.plot.plot_lines}\pysiglinewithargsret{\sphinxcode{pubplots.plot.}\sphinxbfcode{plot\_lines}}{\emph{ax}, \emph{yset}, \emph{lw=2.0}, \emph{dashes=None}, \emph{linestyles='-`}, \emph{colors='tb10'}, \emph{**kwargs}}{}
plot passed data as lines. Note it uses a ziped set  of lists so the shortest
list is the maximum number of plots. TB10 just has 10 colors so it will plot a maximum of 10
lines. For more use `tb20'
\begin{quote}\begin{description}
\item[{Parameters}] \leavevmode
\textbf{ax} : matplotlib.axes object

\textbf{yset} : list
\begin{quote}

list of data to plot like{[}{[}x1array, y1array{]}, {[}x2array, y2array{]}.....{]}.
x1,y1 are arrays or lists of numbers for plotting
\end{quote}

\textbf{lw} : float, optional
\begin{quote}

linewidth
\end{quote}

\textbf{dashes} : None, Bool or list, optional
\begin{quote}

True turns on varying dasshes. or list like {[}{[}7,3{]},{[}9,3,2,3{]}...{]} of dash specifications
can be passed. line.set\_dashes from matplotlib
\end{quote}

\textbf{linestyles} : str or list, optional
\begin{quote}

`-` for continuous lines `--` dashed
\end{quote}

\textbf{colors} : str or list of (r,g,b) tupples
\begin{quote}

Pass a string options are `black', `grey', `tb10', `tb20' and `cb10'(for colorblind people)
\end{quote}

\textbf{**kwargs} : TYPE
\begin{quote}

passed to matplotlib axes.plot()
\end{quote}

\end{description}\end{quote}

\end{fulllineitems}

\index{plot\_lright() (in module pubplots.plot)}

\begin{fulllineitems}
\phantomsection\label{pubplots:pubplots.plot.plot_lright}\pysiglinewithargsret{\sphinxcode{pubplots.plot.}\sphinxbfcode{plot\_lright}}{\emph{ax}, \emph{yset}, \emph{lw=2.0}, \emph{yaxlabel='y2'}, \emph{linestyles='-`}, \emph{color=(0.12156862745098039}, \emph{0.4666666666666667}, \emph{0.7058823529411765)}, \emph{fontsize=18}, \emph{spine=False}, \emph{**kwargs}}{}
plot line to right hand axis. Also colors the right hand labels to of the line and returns
The right hand axis. By default it is the tableau blue color
\begin{quote}\begin{description}
\item[{Parameters}] \leavevmode
\textbf{ax} : matplotlib.axes object

\textbf{yset} : list
\begin{quote}

list of data to plot like{[}{[}x1array, y1array{]}, {[}x2array, y2array{]}.....{]}.
x1,y1 are arrays or pointers to arrays/lists of numbers
\end{quote}

\textbf{lw} : float, optional
\begin{quote}

linewidth
\end{quote}

\textbf{yaxlabel} : str, optional
\begin{quote}

Label to be added to the right hand y axis
\end{quote}

\textbf{linestyles} : str or list, optional
\begin{quote}

`-` for continuous lines `--` dashed
\end{quote}

\textbf{color} : (r,g,b) tupple, or matplotlib color `b'- blue

\textbf{fontsize} : int, optional

\textbf{spine} : bool, optional
\begin{quote}

True - Include the right hand frame spine
\end{quote}

\textbf{**kwargs} : TYPE
\begin{quote}

passed to matplotlib axes.plot()
\end{quote}

\item[{Returns}] \leavevmode
matplotlib.axes object
\begin{quote}

returns the new right hand axes
\end{quote}

\end{description}\end{quote}

\end{fulllineitems}

\index{plot\_lright2() (in module pubplots.plot)}

\begin{fulllineitems}
\phantomsection\label{pubplots:pubplots.plot.plot_lright2}\pysiglinewithargsret{\sphinxcode{pubplots.plot.}\sphinxbfcode{plot\_lright2}}{\emph{ax}, \emph{yset}, \emph{lw=2.0}, \emph{yaxlabel='None'}, \emph{color=(0.8392156862745098}, \emph{0.15294117647058825}, \emph{0.1568627450980392)}, \emph{fontsize=18}, \emph{linestyles='-`}, \emph{spine=False}, \emph{**kwargs}}{}
plot line to displaced right hand axis and returns the axis.
Also colors the right hand labels to that if the
line. By default it is the tableau red color
\begin{quote}\begin{description}
\item[{Parameters}] \leavevmode
\textbf{ax} : matplotlib.axes object

\textbf{yset} : list
\begin{quote}

list of data to plot like{[}{[}x1array, y1array{]}, {[}x2array, y2array{]}.....{]}.
x1,y1 are arrays or pointers to arrays/lists of numbers
\end{quote}

\textbf{lw} : float, optional
\begin{quote}

linewidth
\end{quote}

\textbf{yaxlabel} : str, optional
\begin{quote}

Label to be added to the right hand y axis
\end{quote}

\textbf{linestyles} : str or list, optional
\begin{quote}

`-` for continuous lines `--` dashed
\end{quote}

\textbf{color} : (r,g,b) tupple, or matplotlib color `b'- blue

\textbf{fontsize} : int, optional

\textbf{spine} : bool, optional
\begin{quote}

True - Include the right hand frame spine
\end{quote}

\textbf{**kwargs} : TYPE
\begin{quote}

passed to matplotlib axes.plot()
\end{quote}

\item[{Returns}] \leavevmode
matplotlib.axes object
\begin{quote}

returns the new right hand axes
\end{quote}

\end{description}\end{quote}

\end{fulllineitems}

\index{plot\_scatter() (in module pubplots.plot)}

\begin{fulllineitems}
\phantomsection\label{pubplots:pubplots.plot.plot_scatter}\pysiglinewithargsret{\sphinxcode{pubplots.plot.}\sphinxbfcode{plot\_scatter}}{\emph{ax}, \emph{yset}, \emph{markersize=10}, \emph{fillstyle='full'}, \emph{markers='var'}, \emph{markeredgewidth=0.0}, \emph{colors='tb10'}}{}
plot passed data as a scatter plot. Note it uses a ziped set of lists so the shortest
list is the maximum number of plots. For example colors is currently of length ten.
\begin{quote}\begin{description}
\item[{Parameters}] \leavevmode
\textbf{ax} : matplotlib.axes object

\textbf{yset} : list
\begin{quote}

list of data to plot like{[}{[}x1array, y1array{]}, {[}x2array, y2array{]}.....{]}.
x1,y1 are arrays or pointers to arrays/lists of numbers
\end{quote}

\textbf{markersize} : int, optional, default 10

\textbf{fillstyle} : str, optional, default `full'

\textbf{markers} : str or list of matplotlib markers e.g. {[}'o','s'{]}
\begin{quote}

default is `var' which is a list of matplotlib markers
\end{quote}

\textbf{markeredgewidth} : float, optional, default 0.0

\textbf{colors} : str or list of (r,g,b) tupples
\begin{quote}

Pass a string options are `black', `grey', `tb10', `tb20' and `cb10'(for colorblind people)
\end{quote}

\end{description}\end{quote}

\end{fulllineitems}

\index{plot\_sright() (in module pubplots.plot)}

\begin{fulllineitems}
\phantomsection\label{pubplots:pubplots.plot.plot_sright}\pysiglinewithargsret{\sphinxcode{pubplots.plot.}\sphinxbfcode{plot\_sright}}{\emph{ax}, \emph{yset}, \emph{markersize=8}, \emph{fillstyle='full'}, \emph{markers='var'}, \emph{yaxlabel='y2'}, \emph{color=(0.12156862745098039}, \emph{0.4666666666666667}, \emph{0.7058823529411765)}, \emph{fontsize=18}, \emph{markeredgewidth=0.0}, \emph{spine=False}, \emph{**kwargs}}{}
plot scatter to right hand axis. Also colors the right hand labels to that if the line. By
default it is the tableau blue color
\begin{quote}\begin{description}
\item[{Parameters}] \leavevmode
\textbf{ax} : matplotlib.axes object

\textbf{yset} : list
\begin{quote}

list of data to plot like{[}{[}x1array, y1array{]}, {[}x2array, y2array{]}.....{]}.
x1,y1 are arrays or pointers to arrays/lists of numbers
\end{quote}

\textbf{markersize} : int, optional, default 10

\textbf{fillstyle} : str, optional, default `full'

\textbf{markers} : str or list of matplotlib markers e.g. {[}'o','s'{]}
\begin{quote}

default is `var' which is a list of matplotlib markers
\end{quote}

\textbf{markeredgewidth} : float, optional, default 0.0

\textbf{yaxlabel} : str, optional
\begin{quote}

Label to be added to the right hand y axis
\end{quote}

\textbf{color} : (r,g,b) tupple, or matplotlib color `b'- blue

\textbf{fontsize} : int, optional

\textbf{spine} : bool, optional
\begin{quote}

True - Include the right hand frame spine
\end{quote}

\textbf{**kwargs} : TYPE
\begin{quote}

passed to matplotlib axes.plot()
\end{quote}

\item[{Returns}] \leavevmode
matplotlib.axes object
\begin{quote}

returns the new right hand axes
\end{quote}

\end{description}\end{quote}

\end{fulllineitems}

\index{plot\_sright2() (in module pubplots.plot)}

\begin{fulllineitems}
\phantomsection\label{pubplots:pubplots.plot.plot_sright2}\pysiglinewithargsret{\sphinxcode{pubplots.plot.}\sphinxbfcode{plot\_sright2}}{\emph{ax}, \emph{yset}, \emph{markersize=8}, \emph{fillstyle='full'}, \emph{markers='var'}, \emph{yaxlabel='y3'}, \emph{color=(0.8392156862745098}, \emph{0.15294117647058825}, \emph{0.1568627450980392)}, \emph{fontsize=18}, \emph{markeredgewidth=0.0}, \emph{spine=False}, \emph{**kwargs}}{}
plot scatter to right hand axis. Also colors the right hand labels to that if the line. By
default it is the tableau blue color
\begin{quote}\begin{description}
\item[{Parameters}] \leavevmode
\textbf{ax} : matplotlib.axes object

\textbf{yset} : list
\begin{quote}

list of data to plot like{[}{[}x1array, y1array{]}, {[}x2array, y2array{]}.....{]}.
x1,y1 are arrays or pointers to arrays/lists of numbers
\end{quote}

\textbf{markersize} : int, optional, default 10

\textbf{fillstyle} : str, optional, default `full'

\textbf{markers} : str or list of matplotlib markers e.g. {[}'o','s'{]}
\begin{quote}

default is `var' which is a list of matplotlib markers
\end{quote}

\textbf{markeredgewidth} : float, optional, default 0.0

\textbf{yaxlabel} : str, optional
\begin{quote}

Label to be added to the right hand y axis
\end{quote}

\textbf{color} : (r,g,b) tupple, or matplotlib color `b'- blue

\textbf{fontsize} : int, optional

\textbf{spine} : bool, optional
\begin{quote}

True - Include the right hand frame spine
\end{quote}

\textbf{**kwargs} : TYPEV
\begin{quote}

passed to matplotlib axes.plot()
\end{quote}

\item[{Returns}] \leavevmode
matplotlib.axes object
\begin{quote}

returns the new right hand axes
\end{quote}

\end{description}\end{quote}

\end{fulllineitems}

\index{quick\_modern() (in module pubplots.plot)}

\begin{fulllineitems}
\phantomsection\label{pubplots:pubplots.plot.quick_modern}\pysiglinewithargsret{\sphinxcode{pubplots.plot.}\sphinxbfcode{quick\_modern}}{\emph{ax}, \emph{plotdata}, \emph{figsize=(9}, \emph{6)}, \emph{scatter=False}, \emph{rscatter=False}, \emph{grid=True}, \emph{r2scatter=False}, \emph{at\_x=None}, \emph{label=True}, \emph{fontsize=18}}{}
Make a modern style plot from a PlotData object
\begin{quote}\begin{description}
\item[{Parameters}] \leavevmode
\textbf{ax} : matplotlib.axes object

\textbf{plotdata} : PlotData object see plotdata from pubplots.
\begin{quote}

Holds the information of what data to plot and axislabels line labels etc.
\end{quote}

\textbf{figsize} : tuple, optional
\begin{quote}

(9,6) is the default
\end{quote}

\textbf{scatter} : bool, optional
\begin{quote}

True use markers instead of lines
\end{quote}

\textbf{rscatter} : bool, optional
\begin{quote}

USe markers instead of lines on the right hand axes
\end{quote}

\textbf{grid} : bool, optional
\begin{quote}

Turn on grid for primerary axes
\end{quote}

\textbf{r2scatter} : bool, optional
\begin{quote}

USe markers instead of lines on the right hand axes
\end{quote}

\textbf{at\_x} : None, optional
\begin{quote}

list of x\_positions for the labels on the primerary axes
\end{quote}

\textbf{label} : bool, optional
\begin{quote}

True labels the lines
\end{quote}

\textbf{fontsize} : int, optional

\item[{Returns}] \leavevmode
\textbf{r1,r2} : matplotlib.axes objects
\begin{quote}

returns the new right hand axes, None, None if nothing is plotted to the right hand axes
\end{quote}

\end{description}\end{quote}

\end{fulllineitems}

\index{quick\_old\_hat() (in module pubplots.plot)}

\begin{fulllineitems}
\phantomsection\label{pubplots:pubplots.plot.quick_old_hat}\pysiglinewithargsret{\sphinxcode{pubplots.plot.}\sphinxbfcode{quick\_old\_hat}}{\emph{ax}, \emph{plotdata}, \emph{figsize=(9}, \emph{6)}, \emph{scatter=False}, \emph{rscatter=False}, \emph{r2scatter=False}, \emph{at\_x=None}, \emph{label=True}, \emph{fontsize=18}, \emph{dashes=False}}{}
Make a modern style plot from a PlotData object
\begin{quote}\begin{description}
\item[{Parameters}] \leavevmode
\textbf{ax} : matplotlib.axes object

\textbf{plotdata} : PlotData object see plotdata from pubplots.
\begin{quote}

Holds the information of what data to plot and axislabels line labels etc.
\end{quote}

\textbf{figsize} : tuple, optional
\begin{quote}

(9,6) is the default
\end{quote}

\textbf{scatter} : bool, optional
\begin{quote}

True use markers instead of lines
\end{quote}

\textbf{rscatter} : bool, optional
\begin{quote}

USe markers instead of lines on the right hand axes
\end{quote}

\textbf{grid} : bool, optional
\begin{quote}

Turn on grid for primerary axes
\end{quote}

\textbf{r2scatter} : bool, optional
\begin{quote}

USe markers instead of lines on the right hand axes
\end{quote}

\textbf{at\_x} : None, optional
\begin{quote}

list of x\_positions for the labels on the primerary axes
\end{quote}

\textbf{label} : bool, optional
\begin{quote}

True labels the lines
\end{quote}

\textbf{fontsize} : int, optional

\textbf{dashes} : None, optional
\begin{quote}

True - adds varying dashes
\end{quote}

\item[{Returns}] \leavevmode
\textbf{r1,r2} : matplotlib.axes objects
\begin{quote}

returns the new right hand axes, None, None if nothing is plotted to the right hand axes
\end{quote}

\end{description}\end{quote}

\end{fulllineitems}

\index{quick\_semimodern() (in module pubplots.plot)}

\begin{fulllineitems}
\phantomsection\label{pubplots:pubplots.plot.quick_semimodern}\pysiglinewithargsret{\sphinxcode{pubplots.plot.}\sphinxbfcode{quick\_semimodern}}{\emph{ax}, \emph{plotdata}, \emph{figsize=(9}, \emph{6)}, \emph{scatter=False}, \emph{rscatter=False}, \emph{grid=True}, \emph{r2scatter=False}, \emph{at\_x=None}, \emph{label=True}, \emph{fontsize=18}}{}
Make a modern style plot from a PlotData object
\begin{quote}\begin{description}
\item[{Parameters}] \leavevmode
\textbf{ax} : matplotlib.axes object

\textbf{plotdata} : PlotData object see plotdata from pubplots.
\begin{quote}

Holds the information of what data to plot and axislabels line labels etc.
\end{quote}

\textbf{figsize} : tuple, optional
\begin{quote}

(9,6) is the default
\end{quote}

\textbf{scatter} : bool, optional
\begin{quote}

True use markers instead of lines
\end{quote}

\textbf{rscatter} : bool, optional
\begin{quote}

USe markers instead of lines on the right hand axes
\end{quote}

\textbf{grid} : bool, optional
\begin{quote}

Turn on grid for primerary axes
\end{quote}

\textbf{r2scatter} : bool, optional
\begin{quote}

USe markers instead of lines on the right hand axes
\end{quote}

\textbf{at\_x} : None, optional
\begin{quote}

list of x\_positions for the labels on the primerary axes
\end{quote}

\textbf{label} : bool, optional
\begin{quote}

True labels the lines
\end{quote}

\textbf{fontsize} : int, optional

\item[{Returns}] \leavevmode
\textbf{r1,r2} : matplotlib.axes objects
\begin{quote}

returns the new right hand axes, None, None if nothing is plotted to the right hand axes
\end{quote}

\end{description}\end{quote}

\end{fulllineitems}

\index{remove\_boundary() (in module pubplots.plot)}

\begin{fulllineitems}
\phantomsection\label{pubplots:pubplots.plot.remove_boundary}\pysiglinewithargsret{\sphinxcode{pubplots.plot.}\sphinxbfcode{remove\_boundary}}{\emph{ax}}{}
Remove the spines. Common in modern presentation graphs
\begin{quote}\begin{description}
\item[{Parameters}] \leavevmode
\textbf{ax} : matplotlib.axes object
\begin{quote}

the axes to to have the spines remoced
\end{quote}

\end{description}\end{quote}

\end{fulllineitems}

\index{remove\_right\_top() (in module pubplots.plot)}

\begin{fulllineitems}
\phantomsection\label{pubplots:pubplots.plot.remove_right_top}\pysiglinewithargsret{\sphinxcode{pubplots.plot.}\sphinxbfcode{remove\_right\_top}}{\emph{ax}}{}
Remove top and right boundaries
\begin{quote}\begin{description}
\item[{Parameters}] \leavevmode
\textbf{ax} : matplotlib.axes object
\begin{quote}

the axes to to have the spines remoced
\end{quote}

\end{description}\end{quote}

\end{fulllineitems}

\index{save() (in module pubplots.plot)}

\begin{fulllineitems}
\phantomsection\label{pubplots:pubplots.plot.save}\pysiglinewithargsret{\sphinxcode{pubplots.plot.}\sphinxbfcode{save}}{\emph{name='plot'}}{}
save as png and pdf
\begin{quote}\begin{description}
\item[{Parameters}] \leavevmode
\textbf{name} : str, optional
\begin{quote}

file name
\end{quote}

\end{description}\end{quote}

\end{fulllineitems}

\index{semi\_modern\_style() (in module pubplots.plot)}

\begin{fulllineitems}
\phantomsection\label{pubplots:pubplots.plot.semi_modern_style}\pysiglinewithargsret{\sphinxcode{pubplots.plot.}\sphinxbfcode{semi\_modern\_style}}{\emph{ax}, \emph{fontsize=18}, \emph{grid=True}}{}
Bottom and left boundarys. y-grid nicetableau colors
\begin{quote}\begin{description}
\item[{Parameters}] \leavevmode
\textbf{ax} : matplotlib.axes object

\textbf{fontsize} : int, optional

\textbf{grid} : bool, optional

\end{description}\end{quote}

\end{fulllineitems}

\index{set\_colors() (in module pubplots.plot)}

\begin{fulllineitems}
\phantomsection\label{pubplots:pubplots.plot.set_colors}\pysiglinewithargsret{\sphinxcode{pubplots.plot.}\sphinxbfcode{set\_colors}}{\emph{colors}}{}
Take a key and set the colors accordingly, or return the same list if a list is passed
\begin{quote}\begin{description}
\item[{Parameters}] \leavevmode
\textbf{colors} : str or list of (r,g,b) tupples
\begin{quote}

Pass a string options are `black', `grey', `tb10', `tb20' and `cb10'(for colorblind people)
\end{quote}

\item[{Returns}] \leavevmode
list of tupples
\begin{quote}

A list of (r,g,b) tuples
\end{quote}

\end{description}\end{quote}

\end{fulllineitems}

\index{set\_linestyles() (in module pubplots.plot)}

\begin{fulllineitems}
\phantomsection\label{pubplots:pubplots.plot.set_linestyles}\pysiglinewithargsret{\sphinxcode{pubplots.plot.}\sphinxbfcode{set\_linestyles}}{\emph{linestyles}}{}
return a list of linestyles or unchanged list if a list is passed
\begin{quote}\begin{description}
\item[{Parameters}] \leavevmode
\textbf{linestyles} : str or list
\begin{quote}

options `-` or `--`, `..' or, `.-` or a list with variations of these
\end{quote}

\item[{Returns}] \leavevmode
list
\begin{quote}

passed list or makes a list from `-`
\end{quote}

\end{description}\end{quote}

\end{fulllineitems}

\index{set\_markers() (in module pubplots.plot)}

\begin{fulllineitems}
\phantomsection\label{pubplots:pubplots.plot.set_markers}\pysiglinewithargsret{\sphinxcode{pubplots.plot.}\sphinxbfcode{set\_markers}}{\emph{markers}}{}
return a list of markers if a str key is passed
\begin{quote}\begin{description}
\item[{Parameters}] \leavevmode
\textbf{markers} : str or list
\begin{quote}

`var' gives a list of varying markers
\end{quote}

\item[{Returns}] \leavevmode
list
\begin{quote}

list of markers
\end{quote}

\end{description}\end{quote}

\end{fulllineitems}



\subsection{pubplots.colorsmarkers module}
\label{pubplots:module-pubplots.colorsmarkers}\label{pubplots:pubplots-colorsmarkers-module}\index{pubplots.colorsmarkers (module)}
The colors and markers used in pubplots.plot


\subsubsection{Attributes}
\label{pubplots:attributes}\begin{description}
\item[{BLACK}] \leavevmode{[}list of black color tuples{]}
{[}(0.05, 0.05, 0.05){]}*20

\item[{CB10}] \leavevmode{[}list of color tupples{]}
color blind 10 from
\url{http://tableaufriction.blogspot.de/2012/11/finally-you-can-use-tableau-data-colors.html}

\item[{TB10}] \leavevmode{[}list of color tuples{]}
tableau 20 colors from
\url{http://tableaufriction.blogspot.de/2012/11/finally-you-can-use-tableau-data-colors.html}

\item[{TB20}] \leavevmode{[}list of color tuples{]}
tableau 20 colors from
\url{http://tableaufriction.blogspot.de/2012/11/finally-you-can-use-tableau-data-colors.html}

\item[{TB5}] \leavevmode{[}list of color tuples{]}
tableau 5

\item[{GREY}] \leavevmode{[}list of grey tuple colors{]}
{[}(0.34, 0.34, 0.39){]}*20

\item[{PBMK}] \leavevmode{[}list of markers{]}
{[}'o', `s', `v', `p', `\textasciicircum{}', `8', `*', `\textgreater{}', `\textless{}', `x', `+'{]}

\end{description}

pubdashes : list of dash seperations
pubcolors : dictionary of strings to colors
\begin{quote}

pubcolors=\{`tb10':TB10, `tb20': TB20, `tb5':TB5, `cb10':CB10, `black':BLACK, `grey':GREY\}
\end{quote}

publs : dictionary of linestyles
pubmarkers : dictionary of marker lists
\begin{quote}

pubmarkers=\{`var':PBMK,'o':20*{[}'o'{]}, `s':20*{[}'s'{]}, `v': 20*{[}'v'{]}\}
\end{quote}


\section{Examples}
\label{pubplots:examples}

\subsection{1 three y axis}
\label{pubplots:three-y-axis}
\begin{Verbatim}[commandchars=\\\{\}]
\PYG{k+kn}{import} \PYG{n+nn}{matplotlib}\PYG{n+nn}{.}\PYG{n+nn}{pyplot} \PYG{k}{as} \PYG{n+nn}{plt}
\PYG{k+kn}{from} \PYG{n+nn}{pubplots} \PYG{k}{import} \PYG{n}{PlotData}
\PYG{k+kn}{import} \PYG{n+nn}{pubplots}\PYG{n+nn}{.}\PYG{n+nn}{plot} \PYG{k}{as} \PYG{n+nn}{pbt}

\PYG{c+c1}{\PYGZsh{} Load up the data. This is tga data of the change in mass, the temperature and partial pressure}
\PYG{c+c1}{\PYGZsh{} vs. time. so we need 3 y\PYGZhy{}axis, with the data for each specified by ycol,}
\PYG{c+c1}{\PYGZsh{} yrcols (r stands for right y axis) and yr2cols.}
\PYG{c+c1}{\PYGZsh{} Plot data automatically uses the collumn headers from the first row as axis labels.}
\PYG{n}{timedata} \PYG{o}{=} \PYG{n}{PlotData}\PYG{p}{(}\PYG{p}{)}
\PYG{n}{timedata}\PYG{o}{.}\PYG{n}{onefile}\PYG{p}{(}\PYG{l+s+s1}{\PYGZsq{}}\PYG{l+s+s1}{data/timedata.csv}\PYG{l+s+s1}{\PYGZsq{}}\PYG{p}{,} \PYG{n}{xcol}\PYG{o}{=}\PYG{l+m+mi}{0}\PYG{p}{,} \PYG{n}{ycols}\PYG{o}{=}\PYG{p}{[}\PYG{l+m+mi}{1}\PYG{p}{]}\PYG{p}{,} \PYG{n}{yrcols}\PYG{o}{=}\PYG{p}{[}\PYG{l+m+mi}{2}\PYG{p}{]}\PYG{p}{,} \PYG{n}{yr2cols}\PYG{o}{=}\PYG{p}{[}\PYG{l+m+mi}{3}\PYG{p}{]}\PYG{p}{)}


\PYG{n}{fig} \PYG{o}{=} \PYG{n}{plt}\PYG{o}{.}\PYG{n}{figure}\PYG{p}{(}\PYG{n}{figsize}\PYG{o}{=}\PYG{p}{(}\PYG{l+m+mi}{16}\PYG{p}{,} \PYG{l+m+mi}{6}\PYG{p}{)}\PYG{p}{,} \PYG{n}{facecolor}\PYG{o}{=}\PYG{l+s+s1}{\PYGZsq{}}\PYG{l+s+s1}{white}\PYG{l+s+s1}{\PYGZsq{}}\PYG{p}{)}
\PYG{n}{ax} \PYG{o}{=} \PYG{n}{plt}\PYG{o}{.}\PYG{n}{subplot}\PYG{p}{(}\PYG{p}{)}
\PYG{n}{axr}\PYG{p}{,} \PYG{n}{axr2} \PYG{o}{=} \PYG{n}{pbt}\PYG{o}{.}\PYG{n}{quick\PYGZus{}semimodern}\PYG{p}{(}\PYG{n}{ax}\PYG{p}{,} \PYG{n}{timedata}\PYG{p}{,} \PYG{n}{label}\PYG{o}{=}\PYG{k+kc}{False}\PYG{p}{,} \PYG{n}{fontsize}\PYG{o}{=}\PYG{l+m+mi}{18}\PYG{p}{)}
\PYG{c+c1}{\PYGZsh{} set po2 scale to log and adjust the left axis and the xaxis}
\PYG{n}{axr2}\PYG{o}{.}\PYG{n}{set\PYGZus{}yscale}\PYG{p}{(}\PYG{l+s+s1}{\PYGZsq{}}\PYG{l+s+s1}{log}\PYG{l+s+s1}{\PYGZsq{}}\PYG{p}{)}
\PYG{n}{ax}\PYG{o}{.}\PYG{n}{set\PYGZus{}xlim}\PYG{p}{(}\PYG{l+m+mi}{0}\PYG{p}{,} \PYG{l+m+mf}{34.2}\PYG{p}{)}
\PYG{n}{ax}\PYG{o}{.}\PYG{n}{set\PYGZus{}ylim}\PYG{p}{(}\PYG{o}{\PYGZhy{}}\PYG{l+m+mf}{4.2}\PYG{p}{,} \PYG{l+m+mf}{0.2}\PYG{p}{)}
\PYG{c+c1}{\PYGZsh{} Saves as a pdf and a png in a folder/new folder \PYGZsq{}plots\PYGZsq{}, in the working directory}
\PYG{n}{pbt}\PYG{o}{.}\PYG{n}{save}\PYG{p}{(}\PYG{l+s+s1}{\PYGZsq{}}\PYG{l+s+s1}{1\PYGZus{}TGA\PYGZus{}data\PYGZus{}vs\PYGZus{}time}\PYG{l+s+s1}{\PYGZsq{}}\PYG{p}{)}
\PYG{n}{plt}\PYG{o}{.}\PYG{n}{show}\PYG{p}{(}\PYG{p}{)}
\end{Verbatim}

\noindent\sphinxincludegraphics[width=500pt]{{1_TGA_data_vs_time}.png}


\subsection{2 markers}
\label{pubplots:markers}
\begin{Verbatim}[commandchars=\\\{\}]
\PYG{k+kn}{import} \PYG{n+nn}{matplotlib}\PYG{n+nn}{.}\PYG{n+nn}{pyplot} \PYG{k}{as} \PYG{n+nn}{plt}
\PYG{k+kn}{from} \PYG{n+nn}{pubplots} \PYG{k}{import} \PYG{n}{PlotData}
\PYG{k+kn}{import} \PYG{n+nn}{pubplots}\PYG{n+nn}{.}\PYG{n+nn}{plot} \PYG{k}{as} \PYG{n+nn}{pbt}

\PYG{c+c1}{\PYGZsh{} A plot of some reaction rate data of mine for the re\PYGZhy{}oxidation of Ce1\PYGZhy{}xZrxO(2\PYGZhy{}d)}
\PYG{c+c1}{\PYGZsh{} These files are delimted by tabs so that needs to be specified using sep=\PYGZsq{}\PYGZbs{}t\PYGZsq{}}

\PYG{c+c1}{\PYGZsh{} Load kinetic data}
\PYG{n}{pco} \PYG{o}{=} \PYG{n}{PlotData}\PYG{p}{(}\PYG{p}{)}
\PYG{n}{pco}\PYG{o}{.}\PYG{n}{walkandfind}\PYG{p}{(}\PYG{n}{startpath}\PYG{o}{=}\PYG{l+s+s1}{\PYGZsq{}}\PYG{l+s+s1}{data}\PYG{l+s+s1}{\PYGZsq{}}\PYG{p}{,} \PYG{n}{search}\PYG{o}{=}\PYG{l+s+s1}{\PYGZsq{}}\PYG{l+s+s1}{PC\PYGZus{}}\PYG{l+s+s1}{\PYGZsq{}}\PYG{p}{,} \PYG{n}{xaxislabel}\PYG{o}{=}\PYG{l+s+s1}{\PYGZsq{}}\PYG{l+s+s1}{Time [s]}\PYG{l+s+s1}{\PYGZsq{}}\PYG{p}{,}
                \PYG{n}{yaxislabel}\PYG{o}{=}\PYG{l+s+s1}{\PYGZsq{}}\PYG{l+s+s1}{Fraction complete}\PYG{l+s+s1}{\PYGZsq{}}\PYG{p}{,} \PYG{n}{yraxislabel}\PYG{o}{=}\PYG{l+s+s1}{\PYGZsq{}}\PYG{l+s+s1}{Temperature [\PYGZdl{}\PYGZca{}}\PYG{l+s+se}{\PYGZbs{}\PYGZbs{}}\PYG{l+s+s1}{circ\PYGZdl{}C]}\PYG{l+s+s1}{\PYGZsq{}}\PYG{p}{,}
                \PYG{n}{labels}\PYG{o}{=}\PYG{p}{[}\PYG{l+s+s1}{\PYGZsq{}}\PYG{l+s+s1}{x = 0}\PYG{l+s+s1}{\PYGZsq{}}\PYG{p}{,} \PYG{l+s+s1}{\PYGZsq{}}\PYG{l+s+s1}{0.05}\PYG{l+s+s1}{\PYGZsq{}}\PYG{p}{,} \PYG{l+s+s1}{\PYGZsq{}}\PYG{l+s+s1}{0.1}\PYG{l+s+s1}{\PYGZsq{}}\PYG{p}{,} \PYG{l+s+s1}{\PYGZsq{}}\PYG{l+s+s1}{0.2}\PYG{l+s+s1}{\PYGZsq{}}\PYG{p}{,} \PYG{l+s+s1}{\PYGZsq{}}\PYG{l+s+s1}{0.3 }\PYG{l+s+s1}{\PYGZsq{}}\PYG{p}{]}\PYG{p}{,} \PYG{n}{header}\PYG{o}{=}\PYG{k+kc}{None}\PYG{p}{,} \PYG{n}{sep}\PYG{o}{=}\PYG{l+s+s1}{\PYGZsq{}}\PYG{l+s+se}{\PYGZbs{}t}\PYG{l+s+s1}{\PYGZsq{}}\PYG{p}{)}

\PYG{c+c1}{\PYGZsh{} Load temperature data for the right hand axis}
\PYG{n}{rlabel} \PYG{o}{=} \PYG{l+s+s1}{\PYGZsq{}}\PYG{l+s+s1}{T\PYGZdl{}\PYGZus{}}\PYG{l+s+s1}{\PYGZob{}}\PYG{l+s+s1}{\PYGZbs{}}\PYG{l+s+s1}{mathrm}\PYG{l+s+si}{\PYGZob{}max\PYGZcb{}}\PYG{l+s+s1}{\PYGZcb{}\PYGZdl{} \PYGZdl{} }\PYG{l+s+se}{\PYGZbs{}\PYGZbs{}}\PYG{l+s+s1}{approx\PYGZdl{} 860 \PYGZdl{}\PYGZca{}}\PYG{l+s+s1}{\PYGZob{}}\PYG{l+s+s1}{\PYGZbs{}}\PYG{l+s+s1}{circ\PYGZcb{}\PYGZdl{}C \PYGZdl{}}\PYG{l+s+se}{\PYGZbs{}\PYGZbs{}}\PYG{l+s+s1}{rightarrow\PYGZdl{}}\PYG{l+s+s1}{\PYGZsq{}}
\PYG{n}{pco}\PYG{o}{.}\PYG{n}{onefile}\PYG{p}{(}\PYG{l+s+s1}{\PYGZsq{}}\PYG{l+s+s1}{data/t\PYGZus{}T\PYGZus{}ox.txt}\PYG{l+s+s1}{\PYGZsq{}}\PYG{p}{,} \PYG{n}{ycols}\PYG{o}{=}\PYG{p}{[}\PYG{p}{]}\PYG{p}{,} \PYG{n}{yrcols}\PYG{o}{=}\PYG{p}{[}\PYG{l+m+mi}{1}\PYG{p}{]}\PYG{p}{,}
            \PYG{n}{yrlabels}\PYG{o}{=}\PYG{p}{[}\PYG{n}{rlabel}\PYG{p}{]}\PYG{p}{,} \PYG{n}{header}\PYG{o}{=}\PYG{k+kc}{None}\PYG{p}{,} \PYG{n}{sep}\PYG{o}{=}\PYG{l+s+s1}{\PYGZsq{}}\PYG{l+s+se}{\PYGZbs{}t}\PYG{l+s+s1}{\PYGZsq{}}\PYG{p}{)}

\PYG{c+c1}{\PYGZsh{} In this case the pbt.quick\PYGZus{}modern() method sees that there is only one data set for the}
\PYG{c+c1}{\PYGZsh{} right axes and colors it grey, and then uses colors for the left axes}
\PYG{n}{fig} \PYG{o}{=} \PYG{n}{plt}\PYG{o}{.}\PYG{n}{figure}\PYG{p}{(}\PYG{n}{figsize}\PYG{o}{=}\PYG{p}{(}\PYG{l+m+mf}{8.5}\PYG{p}{,} \PYG{l+m+mi}{6}\PYG{p}{)}\PYG{p}{,} \PYG{n}{facecolor}\PYG{o}{=}\PYG{l+s+s1}{\PYGZsq{}}\PYG{l+s+s1}{white}\PYG{l+s+s1}{\PYGZsq{}}\PYG{p}{)}
\PYG{n}{ax} \PYG{o}{=} \PYG{n}{plt}\PYG{o}{.}\PYG{n}{subplot}\PYG{p}{(}\PYG{l+m+mi}{111}\PYG{p}{)}
\PYG{n}{axr} \PYG{o}{=} \PYG{n}{pbt}\PYG{o}{.}\PYG{n}{quick\PYGZus{}modern}\PYG{p}{(}\PYG{n}{ax}\PYG{p}{,} \PYG{n}{pco}\PYG{p}{,} \PYG{n}{at\PYGZus{}x}\PYG{o}{=}\PYG{p}{[}\PYG{l+m+mi}{60}\PYG{p}{,} \PYG{l+m+mi}{90}\PYG{p}{,} \PYG{l+m+mi}{120}\PYG{p}{,} \PYG{l+m+mi}{140}\PYG{p}{,} \PYG{l+m+mi}{160}\PYG{p}{]}\PYG{p}{,} \PYG{n}{scatter}\PYG{o}{=}\PYG{k+kc}{True}\PYG{p}{)}
\PYG{n}{ax}\PYG{o}{.}\PYG{n}{set\PYGZus{}xlim}\PYG{p}{(}\PYG{l+m+mi}{0}\PYG{p}{,} \PYG{l+m+mi}{250}\PYG{p}{)}
\PYG{n}{ax}\PYG{o}{.}\PYG{n}{set\PYGZus{}ylim}\PYG{p}{(}\PYG{l+m+mi}{0}\PYG{p}{,} \PYG{l+m+mi}{1}\PYG{p}{)}
\PYG{n}{plt}\PYG{o}{.}\PYG{n}{tight\PYGZus{}layout}\PYG{p}{(}\PYG{p}{)}
\PYG{n}{plt}\PYG{o}{.}\PYG{n}{show}\PYG{p}{(}\PYG{p}{)}
\end{Verbatim}

\noindent\sphinxincludegraphics[width=400pt]{{2_scatter}.png}


\subsection{3 double y axis no grid and grid}
\label{pubplots:double-y-axis-no-grid-and-grid}
\begin{Verbatim}[commandchars=\\\{\}]
\PYG{k+kn}{import} \PYG{n+nn}{matplotlib}\PYG{n+nn}{.}\PYG{n+nn}{pyplot} \PYG{k}{as} \PYG{n+nn}{plt}
\PYG{k+kn}{from} \PYG{n+nn}{pubplots} \PYG{k}{import} \PYG{n}{PlotData}
\PYG{k+kn}{import} \PYG{n+nn}{pubplots}\PYG{n+nn}{.}\PYG{n+nn}{plot} \PYG{k}{as} \PYG{n+nn}{pbt}

\PYG{c+c1}{\PYGZsh{} This file shows a plot with a scatter to the right hand axes}


\PYG{n}{reddata} \PYG{o}{=} \PYG{n}{PlotData}\PYG{p}{(}\PYG{p}{)}
\PYG{n}{reddata}\PYG{o}{.}\PYG{n}{onefile}\PYG{p}{(}\PYG{l+s+s1}{\PYGZsq{}}\PYG{l+s+s1}{data/reduction.txt}\PYG{l+s+s1}{\PYGZsq{}}\PYG{p}{,} \PYG{n}{xcol}\PYG{o}{=}\PYG{l+m+mi}{0}\PYG{p}{,} \PYG{n}{ycols}\PYG{o}{=}\PYG{p}{[}\PYG{l+m+mi}{2}\PYG{p}{]}\PYG{p}{,} \PYG{n}{yrcols}\PYG{o}{=}\PYG{p}{[}\PYG{l+m+mi}{1}\PYG{p}{]}\PYG{p}{,}
                \PYG{n}{xaxislabel}\PYG{o}{=}\PYG{l+s+s1}{\PYGZsq{}}\PYG{l+s+s1}{Time [s]}\PYG{l+s+s1}{\PYGZsq{}}\PYG{p}{,} \PYG{n}{yaxislabel}\PYG{o}{=}\PYG{l+s+s1}{\PYGZsq{}}\PYG{l+s+s1}{Temperature [\PYGZdl{}\PYGZca{}}\PYG{l+s+s1}{\PYGZbs{}}\PYG{l+s+s1}{circ\PYGZdl{}C]}\PYG{l+s+s1}{\PYGZsq{}}\PYG{p}{,}
                \PYG{n}{yraxislabel}\PYG{o}{=}\PYG{l+s+s1}{\PYGZsq{}}\PYG{l+s+s1}{Pressure [Pa]}\PYG{l+s+s1}{\PYGZsq{}}\PYG{p}{)}

\PYG{c+c1}{\PYGZsh{} Supose we want lines plotted to the left and markers to the right}
\PYG{c+c1}{\PYGZsh{} And no grid}
\PYG{n}{fig} \PYG{o}{=} \PYG{n}{plt}\PYG{o}{.}\PYG{n}{figure}\PYG{p}{(}\PYG{n}{figsize}\PYG{o}{=}\PYG{p}{(}\PYG{l+m+mi}{8}\PYG{p}{,} \PYG{l+m+mi}{6}\PYG{p}{)}\PYG{p}{,} \PYG{n}{facecolor}\PYG{o}{=}\PYG{l+s+s1}{\PYGZsq{}}\PYG{l+s+s1}{white}\PYG{l+s+s1}{\PYGZsq{}}\PYG{p}{)}
\PYG{n}{ax} \PYG{o}{=} \PYG{n}{plt}\PYG{o}{.}\PYG{n}{subplot}\PYG{p}{(}\PYG{l+m+mi}{111}\PYG{p}{)}
\PYG{n}{axr} \PYG{o}{=} \PYG{n}{pbt}\PYG{o}{.}\PYG{n}{quick\PYGZus{}modern}\PYG{p}{(}\PYG{n}{ax}\PYG{p}{,} \PYG{n}{reddata}\PYG{p}{,} \PYG{n}{rscatter}\PYG{o}{=}\PYG{k+kc}{True}\PYG{p}{,} \PYG{n}{grid}\PYG{o}{=}\PYG{k+kc}{False}\PYG{p}{)}
\PYG{n}{pbt}\PYG{o}{.}\PYG{n}{save}\PYG{p}{(}\PYG{l+s+s1}{\PYGZsq{}}\PYG{l+s+s1}{3\PYGZus{}reduction}\PYG{l+s+s1}{\PYGZsq{}}\PYG{p}{)}
\PYG{n}{plt}\PYG{o}{.}\PYG{n}{show}\PYG{p}{(}\PYG{p}{)}

\PYG{c+c1}{\PYGZsh{} Plot 2}
\PYG{n}{oxdata} \PYG{o}{=} \PYG{n}{PlotData}\PYG{p}{(}\PYG{p}{)}
\PYG{n}{oxdata}\PYG{o}{.}\PYG{n}{onefile}\PYG{p}{(}\PYG{l+s+s1}{\PYGZsq{}}\PYG{l+s+s1}{data/oxidation.txt}\PYG{l+s+s1}{\PYGZsq{}}\PYG{p}{,} \PYG{n}{xcol}\PYG{o}{=}\PYG{l+m+mi}{0}\PYG{p}{,} \PYG{n}{ycols}\PYG{o}{=}\PYG{p}{[}\PYG{l+m+mi}{2}\PYG{p}{]}\PYG{p}{,} \PYG{n}{yrcols}\PYG{o}{=}\PYG{p}{[}\PYG{l+m+mi}{1}\PYG{p}{]}\PYG{p}{,} \PYG{n}{xaxislabel}\PYG{o}{=}\PYG{l+s+s1}{\PYGZsq{}}\PYG{l+s+s1}{Time [s]}\PYG{l+s+s1}{\PYGZsq{}}\PYG{p}{,}
               \PYG{n}{yaxislabel}\PYG{o}{=}\PYG{l+s+s1}{\PYGZsq{}}\PYG{l+s+s1}{Temperature [\PYGZdl{}\PYGZca{}}\PYG{l+s+s1}{\PYGZbs{}}\PYG{l+s+s1}{circ\PYGZdl{}C]}\PYG{l+s+s1}{\PYGZsq{}}\PYG{p}{,} \PYG{n}{yraxislabel}\PYG{o}{=}\PYG{l+s+s1}{\PYGZsq{}}\PYG{l+s+s1}{Pressure [Pa]}\PYG{l+s+s1}{\PYGZsq{}}\PYG{p}{)}

\PYG{c+c1}{\PYGZsh{} Make an old style graph}
\PYG{n}{fig} \PYG{o}{=} \PYG{n}{plt}\PYG{o}{.}\PYG{n}{figure}\PYG{p}{(}\PYG{n}{figsize}\PYG{o}{=}\PYG{p}{(}\PYG{l+m+mi}{8}\PYG{p}{,} \PYG{l+m+mi}{6}\PYG{p}{)}\PYG{p}{,} \PYG{n}{facecolor}\PYG{o}{=}\PYG{l+s+s1}{\PYGZsq{}}\PYG{l+s+s1}{white}\PYG{l+s+s1}{\PYGZsq{}}\PYG{p}{)}
\PYG{n}{ax} \PYG{o}{=} \PYG{n}{plt}\PYG{o}{.}\PYG{n}{subplot}\PYG{p}{(}\PYG{l+m+mi}{111}\PYG{p}{)}
\PYG{n}{axr} \PYG{o}{=} \PYG{n}{pbt}\PYG{o}{.}\PYG{n}{quick\PYGZus{}modern}\PYG{p}{(}\PYG{n}{ax}\PYG{p}{,} \PYG{n}{oxdata}\PYG{p}{,} \PYG{n}{rscatter}\PYG{o}{=}\PYG{k+kc}{True}\PYG{p}{)}
\PYG{n}{pbt}\PYG{o}{.}\PYG{n}{save}\PYG{p}{(}\PYG{l+s+s1}{\PYGZsq{}}\PYG{l+s+s1}{3\PYGZus{}oxidation}\PYG{l+s+s1}{\PYGZsq{}}\PYG{p}{)}
\PYG{n}{plt}\PYG{o}{.}\PYG{n}{show}\PYG{p}{(}\PYG{p}{)}
\end{Verbatim}

\noindent\sphinxincludegraphics[width=400pt]{{3_reduction}.png}

\noindent\sphinxincludegraphics[width=400pt]{{3_oxidation}.png}


\subsection{4 inset}
\label{pubplots:inset}
\begin{Verbatim}[commandchars=\\\{\}]
\PYG{k+kn}{import} \PYG{n+nn}{matplotlib}\PYG{n+nn}{.}\PYG{n+nn}{pyplot} \PYG{k}{as} \PYG{n+nn}{plt}
\PYG{k+kn}{from} \PYG{n+nn}{pubplots} \PYG{k}{import} \PYG{n}{PlotData}
\PYG{k+kn}{import} \PYG{n+nn}{pubplots}\PYG{n+nn}{.}\PYG{n+nn}{plot} \PYG{k}{as} \PYG{n+nn}{pbt}

\PYG{n}{kin} \PYG{o}{=} \PYG{n}{PlotData}\PYG{p}{(}\PYG{p}{)}
\PYG{c+c1}{\PYGZsh{} Use the walk and find method to find the fit data}
\PYG{n}{kin}\PYG{o}{.}\PYG{n}{walkandfind}\PYG{p}{(}\PYG{n}{startpath}\PYG{o}{=}\PYG{l+s+s1}{\PYGZsq{}}\PYG{l+s+s1}{data}\PYG{l+s+s1}{\PYGZsq{}}\PYG{p}{,} \PYG{n}{search}\PYG{o}{=}\PYG{l+s+s1}{\PYGZsq{}}\PYG{l+s+s1}{kin}\PYG{l+s+s1}{\PYGZsq{}}\PYG{p}{,} \PYG{n}{xcol}\PYG{o}{=}\PYG{l+m+mi}{0}\PYG{p}{,} \PYG{n}{ycols}\PYG{o}{=}\PYG{p}{[}\PYG{l+m+mi}{1}\PYG{p}{]}\PYG{p}{,}
                \PYG{n}{labels}\PYG{o}{=}\PYG{p}{[}\PYG{l+s+s1}{\PYGZsq{}}\PYG{l+s+s1}{200 \PYGZdl{}\PYGZca{}}\PYG{l+s+s1}{\PYGZbs{}}\PYG{l+s+s1}{circ\PYGZdl{}C}\PYG{l+s+s1}{\PYGZsq{}}\PYG{p}{,} \PYG{l+s+s1}{\PYGZsq{}}\PYG{l+s+s1}{250}\PYG{l+s+s1}{\PYGZsq{}}\PYG{p}{,} \PYG{l+s+s1}{\PYGZsq{}}\PYG{l+s+s1}{300}\PYG{l+s+s1}{\PYGZsq{}}\PYG{p}{,} \PYG{l+s+s1}{\PYGZsq{}}\PYG{l+s+s1}{350}\PYG{l+s+s1}{\PYGZsq{}}\PYG{p}{,} \PYG{l+s+s1}{\PYGZsq{}}\PYG{l+s+s1}{400}\PYG{l+s+s1}{\PYGZsq{}}\PYG{p}{,} \PYG{l+s+s1}{\PYGZsq{}}\PYG{l+s+s1}{500}\PYG{l+s+s1}{\PYGZsq{}}\PYG{p}{]}\PYG{p}{,}
                \PYG{n}{xaxislabel}\PYG{o}{=}\PYG{l+s+s1}{\PYGZsq{}}\PYG{l+s+s1}{Time [min]}\PYG{l+s+s1}{\PYGZsq{}}\PYG{p}{,} \PYG{n}{yaxislabel}\PYG{o}{=}\PYG{l+s+s1}{\PYGZsq{}}\PYG{l+s+s1}{Fraction complete}\PYG{l+s+s1}{\PYGZsq{}}\PYG{p}{,} \PYG{n}{sep}\PYG{o}{=}\PYG{l+s+s1}{\PYGZsq{}}\PYG{l+s+s1}{ }\PYG{l+s+s1}{\PYGZsq{}}\PYG{p}{)}


\PYG{n}{fig} \PYG{o}{=} \PYG{n}{plt}\PYG{o}{.}\PYG{n}{figure}\PYG{p}{(}\PYG{n}{figsize}\PYG{o}{=}\PYG{p}{(}\PYG{l+m+mi}{8}\PYG{p}{,} \PYG{l+m+mi}{6}\PYG{p}{)}\PYG{p}{,} \PYG{n}{facecolor}\PYG{o}{=}\PYG{l+s+s1}{\PYGZsq{}}\PYG{l+s+s1}{white}\PYG{l+s+s1}{\PYGZsq{}}\PYG{p}{)}
\PYG{n}{ax} \PYG{o}{=} \PYG{n}{plt}\PYG{o}{.}\PYG{n}{subplot}\PYG{p}{(}\PYG{p}{)}
\PYG{n}{pbt}\PYG{o}{.}\PYG{n}{quick\PYGZus{}semimodern}\PYG{p}{(}\PYG{n}{ax}\PYG{p}{,} \PYG{n}{kin}\PYG{p}{,} \PYG{n}{at\PYGZus{}x}\PYG{o}{=}\PYG{p}{[}\PYG{l+m+mf}{2.3}\PYG{p}{,} \PYG{l+m+mf}{1.5}\PYG{p}{,} \PYG{l+m+mf}{1.5}\PYG{p}{,} \PYG{l+m+mf}{0.7}\PYG{p}{,} \PYG{l+m+mf}{0.5}\PYG{p}{,} \PYG{l+m+mf}{0.58}\PYG{p}{]}\PYG{p}{)}
\PYG{n}{ax}\PYG{o}{.}\PYG{n}{set\PYGZus{}xlim}\PYG{p}{(}\PYG{l+m+mi}{0}\PYG{p}{,} \PYG{l+m+mf}{2.2}\PYG{p}{)}
\PYG{n}{ax}\PYG{o}{.}\PYG{n}{set\PYGZus{}ylim}\PYG{p}{(}\PYG{l+m+mi}{0}\PYG{p}{,} \PYG{l+m+mf}{1.05}\PYG{p}{)}

\PYG{c+c1}{\PYGZsh{} Remove the two fastest reactions for the inset plot}
\PYG{n}{kin}\PYG{o}{.}\PYG{n}{yset}\PYG{o}{.}\PYG{n}{pop}\PYG{p}{(}\PYG{p}{)}
\PYG{n}{kin}\PYG{o}{.}\PYG{n}{yset}\PYG{o}{.}\PYG{n}{pop}\PYG{p}{(}\PYG{p}{)}
\PYG{n}{axin} \PYG{o}{=} \PYG{n}{pbt}\PYG{o}{.}\PYG{n}{inset\PYGZus{}plot}\PYG{p}{(}\PYG{n}{fig}\PYG{p}{,} \PYG{n}{ax}\PYG{p}{,} \PYG{n}{kin}\PYG{o}{.}\PYG{n}{yset}\PYG{p}{,} \PYG{n}{xlabel}\PYG{o}{=}\PYG{l+s+s1}{\PYGZsq{}}\PYG{l+s+s1}{Time [min]}\PYG{l+s+s1}{\PYGZsq{}}\PYG{p}{,} \PYG{n}{ylabel}\PYG{o}{=}\PYG{l+s+s1}{\PYGZsq{}}\PYG{l+s+s1}{Fraction complete}\PYG{l+s+s1}{\PYGZsq{}}\PYG{p}{,}
                      \PYG{n}{label}\PYG{o}{=}\PYG{k+kc}{True}\PYG{p}{,} \PYG{n}{at\PYGZus{}x}\PYG{o}{=}\PYG{p}{[}\PYG{l+m+mi}{30}\PYG{p}{,} \PYG{l+m+mi}{15}\PYG{p}{,} \PYG{l+m+mi}{15}\PYG{p}{,} \PYG{l+m+mi}{15}\PYG{p}{]}\PYG{p}{,} \PYG{n}{labels}\PYG{o}{=}\PYG{n}{kin}\PYG{o}{.}\PYG{n}{labels}\PYG{p}{,} \PYG{n}{style}\PYG{o}{=}\PYG{l+s+s1}{\PYGZsq{}}\PYG{l+s+s1}{semimodern}\PYG{l+s+s1}{\PYGZsq{}}\PYG{p}{)}
\PYG{n}{axin}\PYG{o}{.}\PYG{n}{set\PYGZus{}xlim}\PYG{p}{(}\PYG{l+m+mi}{0}\PYG{p}{,} \PYG{l+m+mi}{40}\PYG{p}{)}
\PYG{n}{axin}\PYG{o}{.}\PYG{n}{set\PYGZus{}ylim}\PYG{p}{(}\PYG{l+m+mi}{0}\PYG{p}{,} \PYG{l+m+mf}{1.05}\PYG{p}{)}
\PYG{n}{pbt}\PYG{o}{.}\PYG{n}{save}\PYG{p}{(}\PYG{l+s+s1}{\PYGZsq{}}\PYG{l+s+s1}{4\PYGZus{}inset\PYGZus{}kinetics}\PYG{l+s+s1}{\PYGZsq{}}\PYG{p}{)}
\PYG{c+c1}{\PYGZsh{} In this case plt.show() does not display correctly. But the saved image does.}
\PYG{n}{plt}\PYG{o}{.}\PYG{n}{show}\PYG{p}{(}\PYG{p}{)}
\end{Verbatim}

\noindent\sphinxincludegraphics[width=400pt]{{4_inset_kinetics}.png}


\subsection{5 scatter fits}
\label{pubplots:scatter-fits}
\begin{Verbatim}[commandchars=\\\{\}]
\PYG{k+kn}{import} \PYG{n+nn}{matplotlib}\PYG{n+nn}{.}\PYG{n+nn}{pyplot} \PYG{k}{as} \PYG{n+nn}{plt}
\PYG{k+kn}{from} \PYG{n+nn}{pubplots} \PYG{k}{import} \PYG{n}{PlotData}
\PYG{k+kn}{import} \PYG{n+nn}{pubplots}\PYG{n+nn}{.}\PYG{n+nn}{plot} \PYG{k}{as} \PYG{n+nn}{pbt}

\PYG{n}{fit1} \PYG{o}{=} \PYG{n}{PlotData}\PYG{p}{(}\PYG{p}{)}
\PYG{c+c1}{\PYGZsh{} Use the walk and find method to find the fit data}
\PYG{n}{fit1}\PYG{o}{.}\PYG{n}{walkandfind}\PYG{p}{(}\PYG{n}{startpath}\PYG{o}{=}\PYG{l+s+s1}{\PYGZsq{}}\PYG{l+s+s1}{data}\PYG{l+s+s1}{\PYGZsq{}}\PYG{p}{,} \PYG{n}{search}\PYG{o}{=}\PYG{l+s+s1}{\PYGZsq{}}\PYG{l+s+s1}{fit}\PYG{l+s+s1}{\PYGZsq{}}\PYG{p}{,}
                 \PYG{n}{labels}\PYG{o}{=}\PYG{p}{[}\PYG{l+s+s1}{\PYGZsq{}}\PYG{l+s+s1}{x=0}\PYG{l+s+s1}{\PYGZsq{}}\PYG{p}{,} \PYG{l+s+s1}{\PYGZsq{}}\PYG{l+s+s1}{0.1}\PYG{l+s+s1}{\PYGZsq{}}\PYG{p}{,} \PYG{l+s+s1}{\PYGZsq{}}\PYG{l+s+s1}{0.2}\PYG{l+s+s1}{\PYGZsq{}}\PYG{p}{,} \PYG{l+s+s1}{\PYGZsq{}}\PYG{l+s+s1}{0.3}\PYG{l+s+s1}{\PYGZsq{}}\PYG{p}{]}\PYG{p}{,} \PYG{n}{header}\PYG{o}{=}\PYG{k+kc}{None}\PYG{p}{)}

\PYG{c+c1}{\PYGZsh{} fit scatered data with np.polyfit, deg is by default 1 \PYGZhy{} a line}
\PYG{n}{fit1}\PYG{o}{.}\PYG{n}{fit}\PYG{p}{(}\PYG{n}{deg}\PYG{o}{=}\PYG{l+m+mi}{1}\PYG{p}{)}

\PYG{c+c1}{\PYGZsh{} Here we use the individual commands rather than the quick plot method}
\PYG{n}{fig} \PYG{o}{=} \PYG{n}{plt}\PYG{o}{.}\PYG{n}{figure}\PYG{p}{(}\PYG{n}{figsize}\PYG{o}{=}\PYG{p}{(}\PYG{l+m+mi}{8}\PYG{p}{,} \PYG{l+m+mi}{6}\PYG{p}{)}\PYG{p}{,} \PYG{n}{facecolor}\PYG{o}{=}\PYG{l+s+s1}{\PYGZsq{}}\PYG{l+s+s1}{white}\PYG{l+s+s1}{\PYGZsq{}}\PYG{p}{)}
\PYG{n}{ax} \PYG{o}{=} \PYG{n}{plt}\PYG{o}{.}\PYG{n}{subplot}\PYG{p}{(}\PYG{l+m+mi}{111}\PYG{p}{)}
\PYG{n}{pbt}\PYG{o}{.}\PYG{n}{modern\PYGZus{}style}\PYG{p}{(}\PYG{n}{ax}\PYG{p}{)}
\PYG{n}{pbt}\PYG{o}{.}\PYG{n}{axis\PYGZus{}labels}\PYG{p}{(}\PYG{n}{ax}\PYG{p}{,} \PYG{l+s+s1}{\PYGZsq{}}\PYG{l+s+s1}{10\PYGZdl{}\PYGZca{}}\PYG{l+s+si}{\PYGZob{}3\PYGZcb{}}\PYG{l+s+s1}{\PYGZdl{}/RT}\PYG{l+s+s1}{\PYGZsq{}}\PYG{p}{,} \PYG{l+s+s1}{\PYGZsq{}}\PYG{l+s+s1}{ln(k)}\PYG{l+s+s1}{\PYGZsq{}}\PYG{p}{)}
\PYG{n}{pbt}\PYG{o}{.}\PYG{n}{plot\PYGZus{}scatter}\PYG{p}{(}\PYG{n}{ax}\PYG{p}{,} \PYG{n}{fit1}\PYG{o}{.}\PYG{n}{yset}\PYG{p}{)}
\PYG{n}{pbt}\PYG{o}{.}\PYG{n}{plot\PYGZus{}lines}\PYG{p}{(}\PYG{n}{ax}\PYG{p}{,} \PYG{n}{fit1}\PYG{o}{.}\PYG{n}{fits}\PYG{p}{)}
\PYG{n}{pbt}\PYG{o}{.}\PYG{n}{label\PYGZus{}lines}\PYG{p}{(}\PYG{n}{ax}\PYG{p}{,} \PYG{n}{fit1}\PYG{o}{.}\PYG{n}{fits}\PYG{p}{,} \PYG{n}{labels}\PYG{o}{=}\PYG{n}{fit1}\PYG{o}{.}\PYG{n}{labels}\PYG{p}{,} \PYG{n}{at\PYGZus{}x}\PYG{o}{=}\PYG{p}{[}\PYG{l+m+mf}{0.18}\PYG{p}{,} \PYG{l+m+mf}{0.17}\PYG{p}{,} \PYG{l+m+mf}{0.157}\PYG{p}{,} \PYG{l+m+mf}{0.133}\PYG{p}{]}\PYG{p}{)}
\PYG{n}{ax}\PYG{o}{.}\PYG{n}{set\PYGZus{}xlim}\PYG{p}{(}\PYG{l+m+mf}{0.125}\PYG{p}{,} \PYG{l+m+mf}{0.35}\PYG{p}{)}
\PYG{n}{pbt}\PYG{o}{.}\PYG{n}{save}\PYG{p}{(}\PYG{l+s+s1}{\PYGZsq{}}\PYG{l+s+s1}{5\PYGZus{}fitscatter}\PYG{l+s+s1}{\PYGZsq{}}\PYG{p}{)}
\PYG{n}{plt}\PYG{o}{.}\PYG{n}{show}\PYG{p}{(}\PYG{p}{)}
\end{Verbatim}

\noindent\sphinxincludegraphics[width=400pt]{{5_fitscatter}.png}


\subsection{6 old hat style}
\label{pubplots:old-hat-style}
\begin{Verbatim}[commandchars=\\\{\}]
\PYG{k+kn}{import} \PYG{n+nn}{matplotlib}\PYG{n+nn}{.}\PYG{n+nn}{pyplot} \PYG{k}{as} \PYG{n+nn}{plt}
\PYG{k+kn}{from} \PYG{n+nn}{pubplots} \PYG{k}{import} \PYG{n}{PlotData}
\PYG{k+kn}{import} \PYG{n+nn}{pubplots}\PYG{n+nn}{.}\PYG{n+nn}{plot} \PYG{k}{as} \PYG{n+nn}{pbt}

\PYG{c+c1}{\PYGZsh{} Data from FactSage for the thermal decomposition of H2O as a function of temperature}
\PYG{n}{therm} \PYG{o}{=} \PYG{n}{PlotData}\PYG{p}{(}\PYG{p}{)}
\PYG{n}{therm}\PYG{o}{.}\PYG{n}{onefile}\PYG{p}{(}\PYG{l+s+s1}{\PYGZsq{}}\PYG{l+s+s1}{data/thermolysis.csv}\PYG{l+s+s1}{\PYGZsq{}}\PYG{p}{,} \PYG{n}{xcol}\PYG{o}{=}\PYG{l+m+mi}{1}\PYG{p}{,} \PYG{n}{ycols}\PYG{o}{=}\PYG{p}{[}\PYG{l+m+mi}{19}\PYG{p}{,} \PYG{l+m+mi}{4}\PYG{p}{,} \PYG{l+m+mi}{5}\PYG{p}{,} \PYG{l+m+mi}{6}\PYG{p}{,} \PYG{l+m+mi}{7}\PYG{p}{,} \PYG{l+m+mi}{9}\PYG{p}{]}\PYG{p}{,}
              \PYG{n}{yaxislabel}\PYG{o}{=}\PYG{l+s+s1}{\PYGZsq{}}\PYG{l+s+s1}{moles of species x}\PYG{l+s+s1}{\PYGZsq{}}\PYG{p}{)}

\PYG{c+c1}{\PYGZsh{} Old hat uses frames by default and black lines using dashes to distinguish them}
\PYG{n}{fig} \PYG{o}{=} \PYG{n}{plt}\PYG{o}{.}\PYG{n}{figure}\PYG{p}{(}\PYG{n}{figsize}\PYG{o}{=}\PYG{p}{(}\PYG{l+m+mi}{8}\PYG{p}{,} \PYG{l+m+mi}{6}\PYG{p}{)}\PYG{p}{,} \PYG{n}{facecolor}\PYG{o}{=}\PYG{l+s+s1}{\PYGZsq{}}\PYG{l+s+s1}{white}\PYG{l+s+s1}{\PYGZsq{}}\PYG{p}{)}
\PYG{n}{ax} \PYG{o}{=} \PYG{n}{plt}\PYG{o}{.}\PYG{n}{subplot}\PYG{p}{(}\PYG{p}{)}
\PYG{n}{pbt}\PYG{o}{.}\PYG{n}{quick\PYGZus{}old\PYGZus{}hat}\PYG{p}{(}\PYG{n}{ax}\PYG{p}{,} \PYG{n}{therm}\PYG{p}{,} \PYG{n}{at\PYGZus{}x}\PYG{o}{=}\PYG{p}{[}\PYG{l+m+mi}{3000}\PYG{p}{,} \PYG{l+m+mi}{3800}\PYG{p}{,} \PYG{l+m+mi}{3470}\PYG{p}{,} \PYG{l+m+mi}{4000}\PYG{p}{,} \PYG{l+m+mi}{3600}\PYG{p}{,} \PYG{l+m+mi}{3000}\PYG{p}{,} \PYG{l+m+mi}{3000}\PYG{p}{]}\PYG{p}{,} \PYG{n}{dashes}\PYG{o}{=}\PYG{k+kc}{True}\PYG{p}{)}
\PYG{n}{ax}\PYG{o}{.}\PYG{n}{set\PYGZus{}xlim}\PYG{p}{(}\PYG{l+m+mi}{2000}\PYG{p}{,} \PYG{l+m+mi}{5000}\PYG{p}{)}
\PYG{n}{ax}\PYG{o}{.}\PYG{n}{set\PYGZus{}ylim}\PYG{p}{(}\PYG{l+m+mi}{0}\PYG{p}{,} \PYG{l+m+mf}{1.0}\PYG{p}{)}
\PYG{n}{pbt}\PYG{o}{.}\PYG{n}{save}\PYG{p}{(}\PYG{l+s+s1}{\PYGZsq{}}\PYG{l+s+s1}{6\PYGZus{}old\PYGZus{}hat\PYGZus{}thermolysis}\PYG{l+s+s1}{\PYGZsq{}}\PYG{p}{)}
\PYG{n}{plt}\PYG{o}{.}\PYG{n}{show}\PYG{p}{(}\PYG{p}{)}
\end{Verbatim}

\noindent\sphinxincludegraphics[width=400pt]{{6_old_hat_thermolysis}.png}


\subsection{7 rotated labels}
\label{pubplots:rotated-labels}
\begin{Verbatim}[commandchars=\\\{\}]
\PYG{k+kn}{import} \PYG{n+nn}{sys}
\PYG{k+kn}{import} \PYG{n+nn}{matplotlib}\PYG{n+nn}{.}\PYG{n+nn}{pyplot} \PYG{k}{as} \PYG{n+nn}{plt}
\PYG{k+kn}{from} \PYG{n+nn}{plotdata} \PYG{k}{import} \PYG{n}{PlotData}
\PYG{k+kn}{import} \PYG{n+nn}{plot} \PYG{k}{as} \PYG{n+nn}{pbt}

\PYG{n}{dg} \PYG{o}{=} \PYG{n}{PlotData}\PYG{p}{(}\PYG{p}{)}
\PYG{n}{dg}\PYG{o}{.}\PYG{n}{walkandfind}\PYG{p}{(}\PYG{n}{search}\PYG{o}{=}\PYG{l+s+s1}{\PYGZsq{}}\PYG{l+s+s1}{Delta}\PYG{l+s+s1}{\PYGZsq{}}\PYG{p}{,} \PYG{n}{labels}\PYG{o}{=}\PYG{p}{[}\PYG{l+s+s1}{\PYGZsq{}}\PYG{l+s+s1}{\PYGZdl{}p\PYGZus{}}\PYG{l+s+s1}{\PYGZbs{}}\PYG{l+s+s1}{mathrm}\PYG{l+s+si}{\PYGZob{}O\PYGZus{}2\PYGZcb{}}\PYG{l+s+s1}{=0.001\PYGZdl{} [bar]}\PYG{l+s+s1}{\PYGZsq{}}\PYG{p}{,} \PYG{l+s+s1}{\PYGZsq{}}\PYG{l+s+s1}{1 bar}\PYG{l+s+s1}{\PYGZsq{}}\PYG{p}{,} \PYG{l+s+s1}{\PYGZsq{}}\PYG{l+s+s1}{6 bar}\PYG{l+s+s1}{\PYGZsq{}}\PYG{p}{]}\PYG{p}{)}


\PYG{n}{fig}\PYG{o}{=}\PYG{n}{plt}\PYG{o}{.}\PYG{n}{figure}\PYG{p}{(}\PYG{n}{figsize}\PYG{o}{=}\PYG{p}{(}\PYG{l+m+mi}{8}\PYG{p}{,}\PYG{l+m+mi}{6}\PYG{p}{)}\PYG{p}{)}
\PYG{n}{ax}\PYG{o}{=}\PYG{n}{plt}\PYG{o}{.}\PYG{n}{subplot}\PYG{p}{(}\PYG{l+m+mi}{111}\PYG{p}{)}
\PYG{n}{pbt}\PYG{o}{.}\PYG{n}{axis\PYGZus{}labels}\PYG{p}{(}\PYG{n}{ax}\PYG{p}{,}\PYG{l+s+s1}{\PYGZsq{}}\PYG{l+s+s1}{Temperature [K]}\PYG{l+s+s1}{\PYGZsq{}}\PYG{p}{,} \PYG{l+s+s1}{\PYGZsq{}}\PYG{l+s+s1}{\PYGZdl{}}\PYG{l+s+s1}{\PYGZbs{}}\PYG{l+s+s1}{Delta G\PYGZdl{} [kJ mol\PYGZdl{}\PYGZca{}}\PYG{l+s+s1}{\PYGZob{}}\PYG{l+s+s1}{\PYGZhy{}1\PYGZcb{}\PYGZdl{}]}\PYG{l+s+s1}{\PYGZsq{}}\PYG{p}{)}
\PYG{n}{pbt}\PYG{o}{.}\PYG{n}{old\PYGZus{}hat\PYGZus{}style}\PYG{p}{(}\PYG{n}{ax}\PYG{p}{)}
\PYG{n}{pbt}\PYG{o}{.}\PYG{n}{plot\PYGZus{}lines}\PYG{p}{(}\PYG{n}{ax}\PYG{p}{,} \PYG{n}{dg}\PYG{o}{.}\PYG{n}{yset}\PYG{p}{,} \PYG{n}{colors}\PYG{o}{=}\PYG{l+s+s1}{\PYGZsq{}}\PYG{l+s+s1}{black}\PYG{l+s+s1}{\PYGZsq{}}\PYG{p}{)}
\PYG{n}{ax}\PYG{o}{.}\PYG{n}{set\PYGZus{}xlim}\PYG{p}{(}\PYG{p}{(}\PYG{l+m+mi}{1010}\PYG{p}{,}\PYG{l+m+mi}{1390}\PYG{p}{)}\PYG{p}{)}
\PYG{n}{ax}\PYG{o}{.}\PYG{n}{set\PYGZus{}ylim}\PYG{p}{(}\PYG{p}{(}\PYG{o}{\PYGZhy{}}\PYG{l+m+mi}{60}\PYG{p}{,} \PYG{l+m+mi}{45}\PYG{p}{)}\PYG{p}{)}
\PYG{c+c1}{\PYGZsh{}rotate the labels to the lines}
\PYG{n}{pbt}\PYG{o}{.}\PYG{n}{label\PYGZus{}lines}\PYG{p}{(}\PYG{n}{ax}\PYG{p}{,} \PYG{n}{dg}\PYG{o}{.}\PYG{n}{yset}\PYG{p}{,} \PYG{n}{at\PYGZus{}x}\PYG{o}{=}\PYG{p}{[}\PYG{l+m+mi}{1200}\PYG{p}{,}\PYG{l+m+mi}{1200}\PYG{p}{,}\PYG{l+m+mi}{1200}\PYG{p}{]}\PYG{p}{,} \PYG{n}{labels}\PYG{o}{=}\PYG{n}{dg}\PYG{o}{.}\PYG{n}{labels}\PYG{p}{,} \PYG{n}{rotation\PYGZus{}on}\PYG{o}{=}\PYG{k+kc}{True}\PYG{p}{,} \PYG{n}{colors}\PYG{o}{=}\PYG{l+s+s1}{\PYGZsq{}}\PYG{l+s+s1}{black}\PYG{l+s+s1}{\PYGZsq{}}\PYG{p}{)}
\PYG{n}{ax}\PYG{o}{.}\PYG{n}{axhline}\PYG{p}{(}\PYG{n}{y}\PYG{o}{=}\PYG{l+m+mi}{0}\PYG{p}{,} \PYG{n}{color}\PYG{o}{=}\PYG{l+s+s1}{\PYGZsq{}}\PYG{l+s+s1}{k}\PYG{l+s+s1}{\PYGZsq{}}\PYG{p}{)}
\PYG{n}{pbt}\PYG{o}{.}\PYG{n}{save}\PYG{p}{(}\PYG{l+s+s1}{\PYGZsq{}}\PYG{l+s+s1}{7\PYGZus{}rotatedlabels}\PYG{l+s+s1}{\PYGZsq{}}\PYG{p}{)}
\PYG{n}{plt}\PYG{o}{.}\PYG{n}{show}\PYG{p}{(}\PYG{p}{)}
\end{Verbatim}

\noindent\sphinxincludegraphics[width=400pt]{{7_rotatedlabels}.png}


\chapter{Indices and tables}
\label{index:indices-and-tables}\begin{itemize}
\item {} 
\DUrole{xref,std,std-ref}{genindex}

\item {} 
\DUrole{xref,std,std-ref}{modindex}

\item {} 
\DUrole{xref,std,std-ref}{search}

\end{itemize}


\renewcommand{\indexname}{Python Module Index}
\begin{theindex}
\def\bigletter#1{{\Large\sffamily#1}\nopagebreak\vspace{1mm}}
\bigletter{p}
\item {\texttt{pubplots.colorsmarkers}}, \pageref{pubplots:module-pubplots.colorsmarkers}
\item {\texttt{pubplots.plot}}, \pageref{pubplots:module-pubplots.plot}
\end{theindex}

\renewcommand{\indexname}{Index}
\printindex
\end{document}
